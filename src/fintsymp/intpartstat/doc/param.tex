\documentclass[11pt]{article}
\usepackage[utf8]{inputenc}
\usepackage[T1]{fontenc}
\usepackage{geometry}                % See geometry.pdf to learn the layout options. There are lots.
\geometry{a4paper}                   % ... or a4paper or a5paper or ... 
%\geometry{landscape}                % Activate for for rotated page geometry
%\usepackage[parfill]{parskip}    % Activate to begin paragraphs with an empty line rather than an indent
\usepackage{graphicx}
\usepackage{amssymb}
\usepackage{epstopdf}
\DeclareGraphicsRule{.tif}{png}{.png}{`convert #1 `dirname #1`/`basename #1 .tif`.png}
\usepackage{tabularx}

\title{INTPARTSTAT}
\author{Micka\"el Gastineau}

\begin{document}
\maketitle


Programme d'int\'egration d'un seul syst\`eme plan\'etaire avec plusieurs particules.

\section{Version s\'equentielle}

\begin{itemize}
 \item Compilation:  
 
make clean
make
 
\item Execution en interactif: 

intplastat.x  ???.par
\item Soumission sur bessel

qsubserial -fastsse4 intpartstat.x  ???.par
\end{itemize}

\section{Version MPI}

\begin{itemize}
 \item Compilation:  
 
 make mpi
 
\item Soumission sur bessel (ici 48 coeurs) : 

qsubmpi 48 -fastsse4 -stdin STDIN/ intpartstat\_mpi.x  ???.par

\item Fusion des fichiers mpi des diff\'erents processeurs (ici, chemin="DATA" et nf\_rad="sim2014XX" )

mergempi.sh DATA sim2014XX

\end{itemize}



\section{Fichiers d'entree}

\subsection{Fichier de param\`etres  intpartstat.par}

\subsubsection*{Namelist lect : Contr\^oles de l'int\'egration}

\begin{tabularx}{\textwidth}{|l|X|}
\hline
Nom du champ& Descriptif \\ \hline 
chemin   & dossier o\`u seront stock\'es les fichiers \\ \hline
 nf\_rad    & radical de tous les fichiers g\'en\'er\'es\\ \hline
 nf\_initext& fichier de conditions initiales des plan\`etes\\ \hline

 ref\_gmsun& Valeur du GM du soleil de r\'ef\'erence \\ 
& 0: valeur issue de la Table 1 de "NOMINAL VALUES FOR SELECTED SOLAR AND PLANETARY QUANTITIES: IAU 2015 RESOLUTION B3"\\
& 1: valeur calculée à partir de la constante de Gauss (k=0.01720209895e0) \\\hline

 int\_type& sch\'ema de l'int\'egrateur (e.g., 'ABA4' ou 'ABAH4' (liste dans doc de intplastat) ) \\ \hline

 if\_orb\_pla &  = 0, l'intégration des planètes est réalisée en même temps que celle des particules.\\
&  = 1, la solution planétaire est donnée par une fonction tabulée : cf. namelist orb\_pla\_tabulee.\\ \hline

 tinit & temps initial (en g\'eneral 0) \\ \hline

 dt& pas de temps de l'int\'egration en ann\'ee \\ \hline

 n\_iter& nombre de pas d'int\'egrations \`a calculer. A la fin de l'int\'egration, le temps final sera  n\_iter*dt ans.\\ \hline

 n\_out & fr\'equence d'\'ecriture des int\'egrales premi\`eres, coordonn\'ees cart\'esiennes et \'el\'ements elliptiques. Il est exprim\'e en nombre de pas d'int\'egrations. Les donn\'ees seront \'ecrites tous les n\_out*dt ann\'ees.
 \\ \hline
 if\_invar & =0 , l'int\'egration se fait dans le rep\`ere actuel. \\
& =1, l'int\'egration se fait dans le plan invariant et les donn\'ees g\'en\'er\'ees sont dans ce plan invariant 
\\ \hline

 if\_int & =0 , les int\'egrales premi\`eres ne sont pas \'ecrites.\\
&=1, les int\'egrales premi\`eres sont \'ecrites dans les fichiers xxx.int. Un seul fichier\\ \hline
 

part\_blocksize & nombre de particules int\'egr\'ees en m\^eme que le systs\`eme plan\'etaire. Pour MPI, c'est aussi le nombre de particules envoy\'ees aux noeuds esclaves.\\ \hline
nf\_initpart & fichier de conditions initiales des particules\\ \hline
\end{tabularx}


\subsubsection*{Namelist orb\_pla\_tabulee : solution planétaire tabulée}
Ce namelist n'est utilisé que si if\_orb\_pla=1. Il requiert que if\_invar=0.

\begin{tabularx}{\textwidth}{|l|X|}
\hline
Nom du champ& Descriptif \\ \hline 
orb\_pla\_tabulee\_coord & type de coordonn\'ees des plan\`etes \\
&= 5:  positions vitesses h\'eliocentriques (x,y,z,vx,vy,vz)

 \\\hline
orb\_pla\_tabulee\_nf & nom du fichier contenant la solution tabulée. 
Le format identique à celui des fichiers ???.car ou ???.ell.\\ \hline
\end{tabularx}


%%%%%%%%%%%%%%%%%%%%%%%%%%%%%%%%%%%%%%
\vspace{0.5cm}
\subsubsection*{Sortie des coordonn\'ees des plan\`etes}
Cela g\'en\`ere les fichiers xxx.ell et xxx.car.

\begin{tabularx}{\textwidth}{|l|X|}
\hline
Nom du champ& Descriptif \\ \hline \hline
 out\_ell\_pla & format des \'el\'ements elliptiques \'ecrites dans les fichiers xxx.ell \\
&1:  elliptiques h\'eliocentriques canoniques\\
&	     CI(1:6) = (a,e,I,M,omega,Omega)\\
&2:  elliptiques h\'eliocentriques non canoniques\\
&	     CI(1:6) = (a,e,I,M,omega,Omega)\\
&3:  elliptiques h\'eliocentriques canoniques\\
&	     CI(1:6) = (a,la,k,h,q,p)\\
&4:  elliptiques h\'eliocentriques non canoniques\\
&	     CI(1:6) = (a,la,k,h,q,p)\\ \hline
 if\_ell\_pla  & =0 , les \'el\'ements elliptiques des plan\`etes ne sont pas \'ecrits.\\
&=1, les \'el\'ements elliptiques  des plan\`etes sont \'ecrits dans les fichiers xxx.ell. Un seul fichier.\\ \hline
 
 if\_car\_pla  &  =0 , les \'el\'ements cart\'esiens (positions/vitesses)  des plan\`etes  ne sont pas \'ecrits.\\
&=1, les \'el\'ements cart\'esiens (positions/vitesses)  des plan\`etes  sont \'ecrits dans les fichiers xxx.car. Un seul fichier.\\ \hline 
 \end{tabularx}

%%%%%%%%%%%%%%%%%%%%%%%%%%%%%%%%%%%%%%
\vspace{0.5cm}
\subsubsection*{Sortie des coordonn\'ees des particules}
Cela g\'en\`ere les fichiers xxx.ell\_part et xxx.car\_part.

\begin{tabularx}{\textwidth}{|l|X|}
\hline
Nom du champ& Descriptif \\ \hline \hline
 out\_ell\_part & format des \'el\'ements elliptiques \'ecrites dans les fichiers xxx.ell\_part \\
&1:  elliptiques h\'eliocentriques canoniques\\
&	     CI(1:6) = (a,e,I,M,omega,Omega)\\
&2:  elliptiques h\'eliocentriques non canoniques\\
&	     CI(1:6) = (a,e,I,M,omega,Omega)\\
&3:  elliptiques h\'eliocentriques canoniques\\
&	     CI(1:6) = (a,la,k,h,q,p)\\
&4:  elliptiques h\'eliocentriques non canoniques\\
&	     CI(1:6) = (a,la,k,h,q,p)\\ \hline
 if\_ell\_part  & =0 , les \'el\'ements elliptiques des particules ne sont pas \'ecrits.\\
&=1, les \'el\'ements elliptiques  des particules sont \'ecrits dans les fichiers xxx.ell\_part.\\ \hline
 
 if\_car\_part  &  =0 , les \'el\'ements cart\'esiens (positions/vitesses)  des particules ne sont pas \'ecrits.\\
&=1, les \'el\'ements cart\'esiens (positions/vitesses)  des particules  sont \'ecrits dans les fichiers xxx.car\_part.\\ \hline 
 \end{tabularx}


%%%%%%%%%%%%%%%%%%%%%%%%%%%%%%%%%%%%%%
\vspace{0.5cm}
\subsubsection*{Calcul des minimum, moyenne et maximum en a,e,I des particules}
Cela g\'en\`ere les fichiers xxx.minmax\_aei\_part.

\begin{tabularx}{\textwidth}{|l|X|}
\hline
Nom du champ& Descriptif \\ \hline \hline
minmax\_aei\_compute &  =0, les minimum, moyenne et maximum en a,e,I ne sont pas calcul\'ees. Tous les autres champs sont ignor\'es.\\
&=1, les minimum, moyenne et maximum en a,e,I sont calcul\'es. Un fichier par processeur.\\ \hline

minmax\_aei\_stepcalc  & fr\'equence de calcul des minimum, moyenne et maximum en a,e,I. Il est exprim\'e en nombre de pas d'int\'egrations. Les donn\'ees seront calcul\'ees tous les minmax\_aei\_stepcalc*dt ann\'ees. \\ \hline

minmax\_aei\_stepout  & Longueur en nombre d'it\'erations sur laquelle on effectue les calculs de minimum, moyenne et maximum en a,e,I. Les minimum, moyenne et maximum en a,e,I sont \'ecrites tous les minmax\_aei\_stepout*dt ann\'ees dans les fichiers minmax\_aei.\\ \hline

minmax\_aei\_elltype  & Type des \'el\'ements elliptiques utilis\'e pour le calcul des  minimum, moyenne et maximum en a,e,I\\ 
&1:  elliptiques h\'eliocentriques canoniques\\
&	     CI(1:6) = (a,e,I,M,omega,Omega)\\
&2:  elliptiques h\'eliocentriques non canoniques\\
&	     CI(1:6) = (a,e,I,M,omega,Omega)\\ \hline

 \end{tabularx}

%%%%%%%%%%%%%%%%%%%%%%%%%%%%%%%%%%%%%%
\vspace{0.5cm}
\subsubsection*{Calcul des minimum, moyenne et maximum en diff\'erence d'\'el\'ements elliptiques : $a_{part}-a_{p(1)}$, $\lambda_{part}-\lambda_{p(1)}$ et $\varpi_{part}-\varpi_{p(1)}$}
Cela g\'en\`ere les fichiers xxx.minmax\_alp\_part. Le calcul est effectu\'e entre chaque particule et la plan\`ete d'indice $p(1)$.

\begin{tabularx}{\textwidth}{|l|X|}
\hline
minmax\_diffalp\_compute  & =0, les minimum, moyenne et maximum en  $a_{part}-a_{p(1)}$, $\lambda_{part}-\lambda_{p(1)}$ et $\varpi_{part}-\varpi_{p(1)}$ ne sont pas calcul\'ees. Tous les autres champs sont ignor\'es.\\
&=1, les minimum, moyenne et maximum en  $a_{part}-a_{p(1)}$, $\lambda_{part}-\lambda_{p(1)}$ et $\varpi_{part}-\varpi_{p(1)}$ sont calcul\'es. Un fichier par processeur.\\ \hline
minmax\_diffalp\_stepcalc   & fr\'equence de calcul des minimum, moyenne et maximum en a,e,I. Il est exprim\'e en nombre de pas d'int\'egrations. Les donn\'ees seront calcul\'ees tous les minmax\_diffalp\_stepcalc*dt ann\'ees.\\ \hline

minmax\_diffalp\_stepout   &Longueur en nombre d'it\'erations sur laquelle on effectue les calculs de minimum, moyenne et maximum en a,e,I. Les minimum, moyenne et maximum en a,e,I sont \'ecrites tous les minmax\_diffalp\_stepout*dt ann\'ees dans les fichiers minmax\_alp. \\ \hline
minmax\_diffalp\_elltype  & Type des \'el\'ements elliptiques utilis\'e pour le calcul des  minimum, moyenne et maximum en $a_{part}-a_{p(1)}$, $\lambda_{part}-\lambda_{p(1)}$ et $\varpi_{part}-\varpi_{p(1)}$ \\
&6:  elliptiques h\'eliocentriques non canoniques\\
&	     CI(1:6) = (a,e,I,$\lambda$,$\varpi$,Omega)\\ \hline
minmax\_diffalp\_pla(1)   & indice de la plan\`ete $p(1)$. Les indices commencent \`a 1.\\ \hline
 \end{tabularx}

%%%%%%%%%%%%%%%%%%%%%%%%%%%%%%%%%%%%%%
\vspace{0.5cm}
\subsubsection*{Analyse en fr\'equence en $a\exp^{\imath\lambda}, k+\imath h, q+\imath p$ pour les particules}
Cela g\'en\`ere les fichiers xxx.naf\_alkhqp\_part ou xxx.naf\_alkh\_part selon la variable naf\_alkhqp\_compute.

\begin{tabularx}{\textwidth}{|l|X|}
\hline
Nom du champ& Descriptif \\ \hline \hline
naf\_alkhqp\_compute &  =0, l'analyse en fr\'equence en $a\exp^{\imath\lambda}, k+\imath h, q+\imath p$ n'est pas calcul\'ee. Tous les autres champs sont ignor\'es.\\
&=1, l'analyse en fr\'equence en $a\exp^{\imath\lambda}, k+\imath h, q+\imath p$ est calcul\'e. Un fichier par processeur avec l'extension naf\_alkhqp\_part.\\ 
&=2, l'analyse en fr\'equence en $a\exp^{\imath\lambda}, k+\imath h$ est calcul\'e (utile pour le cas  plan (q=p=0)). Un fichier par processeur avec l'extension naf\_alkh\_part.\\ \hline

naf\_alkhqp\_stepcalc  & fr\'equence des points utilis\'es pour l'analyse en fr\'equence. Il est exprim\'e en nombre de pas d'int\'egrations. Les entr\'ees de l'analyse en fr\'equence seront calcul\'ees tous les naf\_alkhqp\_stepcalc*dt ann\'ees. \\ \hline

naf\_alkhqp\_stepout  & Longueur en nombre d'it\'erations sur laquelle on effectue l'analyse en fr\'equence. Le r\'esultat de l'analyse en fr\'equence est \'ecrit tous les naf\_alkhqp\_stepout*dt ann\'ees dans les fichiers naf\_alkhqp ou naf\_alkh.\\ \hline

naf\_alkhqp\_elltype  & Type des \'el\'ements elliptiques utilis\'e pour le calcul de l'analyse en fr\'equence\\ 
&3:  elliptiques h\'eliocentriques canoniques\\
&	     CI(1:6) = (a,la,k,h,q,p)\\
&4:  elliptiques h\'eliocentriques non canoniques\\
&	     CI(1:6) = (a,la,k,h,q,p)\\ \hline
naf\_alkhqp\_nterm  & Nombre de termes recherch\'es pour l'analyse en fr\'equence.\\ \hline
naf\_alkhqp\_isec  & =0, la m\'ethode des secantes n'est pas utilis\'ee.\\
&=1, la m\'ethode des secantes est utilis\'ee.\\ \hline
naf\_alkhqp\_iw  & pr\'esence de fen\^etre.\\
&=-1, fenetre exponentielle PHI(T) = $1/CE*EXP(-1/(1-T^2))$ avec CE= 0.22199690808403971891E0\\
&=0, pas de fen\^etre.\\
&= $N>0$ : PHI(T) = CN*(1+COS(PI*T))**N avec CN = $2^N(N!)^2/(2N)!$\\ \hline
naf\_alkhqp\_dtour  & Longueur d'un tour de cadran ( en g\'en\'eral $2\pi$).\\ \hline
naf\_alkhqp\_tol  &  Tol\'erance pour d\'eterminer si deux fr\'equences sont identiques.\\ \hline

 \end{tabularx}


%%%%%%%%%%%%%%%%%%%%%%%%%%%%%%%%%%%%%%
\vspace{0.5cm}
\subsubsection*{Analyse en fr\'equence en $\exp^{\imath(\lambda_{part}-\lambda_{p(1)})}$ et $\exp^{\imath(\varpi_{part}-\varpi_{p(1)})}$}
Cela g\'en\`ere les fichiers xxx.naf\_diffalp\_part. Le calcul est effectué entre chaque particule et la plan\`ete $p(1)$.

\begin{tabularx}{\textwidth}{|l|X|}
\hline
Nom du champ& Descriptif \\ \hline \hline
naf\_diffalp\_compute &=0, l'analyse en fr\'equence en $\exp^{\imath(\lambda_{part}-\lambda_{p(1)})}$ et $\exp^{\imath(\varpi_{part}-\varpi_{p(1)})}$ n'est pas calcul\'ee. Tous les autres champs sont ignor\'es.\\
&=1, l'analyse en fr\'equence en  $\exp^{\imath(\lambda_{part}-\lambda_{p(1)})}$ et $\exp^{\imath(\varpi_{part}-\varpi_{p(1)})}$ est calcul\'e. Un fichier par processeur avec l'extension naf\_diffalp.\\ \hline

naf\_diffalp\_stepcalc  & fr\'equence des points utilis\'es pour l'analyse en fr\'equence. Il est exprim\'e en nombre de pas d'int\'egrations. Les entr\'ees de l'analyse en fr\'equence seront calcul\'ees tous les naf\_diffalp\_stepcalc*dt ann\'ees. \\ \hline

naf\_diffalp\_stepout  & Longueur en nombre d'it\'erations sur laquelle on effectue l'analyse en fr\'equence. Le r\'esultat de l'analyse en fr\'equence est \'ecrit tous les naf\_diffalp\_stepout*dt ann\'ees dans les fichiers naf\_diffalp ou naf\_alkh.\\ \hline

naf\_diffalp\_elltype  & Type des \'el\'ements elliptiques utilis\'es pour le calcul de l'analyse en fr\'equence\\ 
&6:  elliptiques h\'eliocentriques non canoniques\\
&	     CI(1:6) = (a,e,I,$\lambda$,$\varpi$,Omega)\\ \hline
naf\_diffalp\_nterm  & Nombre de termes recherch\'es pour l'analyse en fr\'equence.\\ \hline
naf\_diffalp\_isec  & =0, la m\'ethode des secantes n'est pas utilis\'ee.\\
&=1, la m\'ethode des secantes est utilis\'ee.\\ \hline
naf\_diffalp\_iw  & pr\'esence de fen\^etre.\\
&=-1, fenetre exponentielle PHI(T) = $1/CE*EXP(-1/(1-T^2))$ avec CE= 0.22199690808403971891E0\\
&=0, pas de fen\^etre.\\
&= $N>0$ : PHI(T) = CN*(1+COS(PI*T))**N avec CN = $2^N(N!)^2/(2N)!$\\ \hline
naf\_diffalp\_dtour  & Longueur d'un tour de cadran ( en g\'en\'eral $2\pi$).\\ \hline
naf\_diffalp\_tol  &  Tol\'erance pour d\'eterminer si deux fr\'equences sont identiques.\\ \hline
naf\_diffalp\_pla(1)   & indice de la premi\`ere plan\`ete $p(1)$. Les indices commencent \`a 1.\\ \hline

 \end{tabularx}

%%%%%%%%%%%%%%%%%%%%%%%%%%%%%%%%%%%%%%
\vspace{0.5cm}
\subsubsection*{Analyse en fr\'equence en $a\exp^{\imath\lambda}, k+\imath h, q+\imath p$ pour les plan\`etes}
Cela g\'en\`ere les fichiers xxx.naf\_alkhqp ou xxx.naf\_alkh selon la variable naf\_alkhqp\_pla\_compute.

\begin{tabularx}{\textwidth}{|l|X|}
\hline
Nom du champ& Descriptif \\ \hline \hline
naf\_alkhqp\_pla\_compute &  =0, l'analyse en fr\'equence en $a\exp^{\imath\lambda}, k+\imath h, q+\imath p$ n'est pas calcul\'ee. Tous les autres champs sont ignor\'es.\\
&=1, l'analyse en fr\'equence en $a\exp^{\imath\lambda}, k+\imath h, q+\imath p$ est calcul\'e. Un fichier par processeur avec l'extension naf\_alkhqp.\\ 
&=2, l'analyse en fr\'equence en $a\exp^{\imath\lambda}, k+\imath h$ est calcul\'e (utile pour le cas  plan (q=p=0)). Un fichier par processeur avec l'extension naf\_alkh.\\ \hline

naf\_alkhqp\_pla\_stepcalc  & fr\'equence des points utilis\'es pour l'analyse en fr\'equence. Il est exprim\'e en nombre de pas d'int\'egrations. Les entr\'ees de l'analyse en fr\'equence seront calcul\'ees tous les naf\_alkhqp\_stepcalc*dt ann\'ees. \\ \hline

naf\_alkhqp\_pla\_stepout  & Longueur en nombre d'it\'erations sur laquelle on effectue l'analyse en fr\'equence. Le r\'esultat de l'analyse en fr\'equence est \'ecrit tous les naf\_alkhqp\_stepout*dt ann\'ees dans les fichiers naf\_alkhqp ou naf\_alkh.\\ \hline

naf\_alkhqp\_pla\_elltype  & Type des \'el\'ements elliptiques utilis\'e pour le calcul de l'analyse en fr\'equence\\ 
&3:  elliptiques h\'eliocentriques canoniques\\
&	     CI(1:6) = (a,la,k,h,q,p)\\
&4:  elliptiques h\'eliocentriques non canoniques\\
&	     CI(1:6) = (a,la,k,h,q,p)\\ \hline
naf\_alkhqp\_pla\_nterm  & Nombre de termes recherch\'es pour l'analyse en fr\'equence.\\ \hline
naf\_alkhqp\_pla\_isec  & =0, la m\'ethode des secantes n'est pas utilis\'ee.\\
&=1, la m\'ethode des secantes est utilis\'ee.\\ \hline
naf\_alkhqp\_pla\_iw  & pr\'esence de fen\^etre.\\
&=-1, fenetre exponentielle PHI(T) = $1/CE*EXP(-1/(1-T^2))$ avec CE= 0.22199690808403971891E0\\
&=0, pas de fen\^etre.\\
&= $N>0$ : PHI(T) = CN*(1+COS(PI*T))**N avec CN = $2^N(N!)^2/(2N)!$\\ \hline
naf\_alkhqp\_pla\_dtour  & Longueur d'un tour de cadran ( en g\'en\'eral $2\pi$).\\ \hline
naf\_alkhqp\_pla\_tol  &  Tol\'erance pour d\'eterminer si deux fr\'equences sont identiques.\\ \hline

 \end{tabularx}

%%%%%%%%%%%%%%%%%%%%%%%%%%%%%%%%%%%%%%
\vspace{0.5cm}
\subsubsection*{Contr\^ole de la distance \`a l'\'etoile}
Cela arr\^ete l'int\'egration de la particule si la particule s'approche trop pr\`es ou s'\'eloigne trop de l'\'etoile. 

\begin{tabularx}{\textwidth}{|l|X|}
\hline
Nom du champ& Descriptif \\ \hline \hline
ctrl\_diststar\_compute & =0, le contr\^ole de distance n'est pas r\'ealis\'e. Tous les autres champs sont ignor\'es.\\
&=1, le contr\^ole de distance est r\'ealis\'e.\\  \hline
ctrl\_diststar\_stepcalc & fr\'equence des points de  contr\^ole de distance. Il est exprim\'e en nombre de pas d'int\'egrations. La distance sera v\'erifi\'ee tous les ctrl\_diststar\_stepcalc*dt ann\'ees. \\ \hline
ctrl\_diststar\_distmin & distance minimale en UA \`a l'\'etoile. Si une particule a une distance \`a l'\'etoile inf\'erieure \`a cette valeur, l'int\'egration de celle-ci s'arr\^ete. \\ \hline
ctrl\_diststar\_distmax & distance maximale en UA \`a l'\'etoile. Si une particule a une distance \`a l'\'etoile sup\'erieure \`a cette valeur, l'int\'egration de celle-ci s'arr\^ete. \\ \hline
 \\ \hline
 \end{tabularx}

%%%%%%%%%%%%%%%%%%%%%%%%%%%%%%%%%%%%%%
\vspace{0.5cm}
\subsubsection*{Contr\^ole de la distance aux plan\`etes}
Cela arr\^ete l'int\'egration de la particule si la particule s'approche trop pr\`es d'une plan\`ete. 

\begin{tabularx}{\textwidth}{|l|X|}
\hline
Nom du champ& Descriptif \\ \hline \hline
ctrl\_distpla\_compute & =0, le contr\^ole de distance n'est pas r\'ealis\'e. Tous les autres champs sont ignor\'es.\\
&=1, le contr\^ole de distance est r\'ealis\'e.\\  \hline
ctrl\_distpla\_stepcalc & fr\'equence des points de  contr\^ole de distance. Il est exprim\'e en nombre de pas d'int\'egrations. La distance sera v\'erifi\'ee tous les ctrl\_diststar\_stepcalc*dt ann\'ees. \\ \hline
ctrl\_distpla\_distmin & distance minimale en UA \`a l'\'etoile. Si une particule a une distance \`a l'\'etoile inf\'erieure \`a cette valeur, l'int\'egration de celle-ci s'arr\^ete. \\ \hline
 \end{tabularx}


%%%%%%%%%%%%%%%%%%%%%%%%%%%%%%%%%%%%%%
\subsection{Format du fichier nf\_initext}

Ce fichier contient les conditions initiales (masses et coordonn\'ees) du syst\`eme plan\'etaire. 
Ce fichier stocke un syst\`eme plan\'etaire par ligne.
Le fichier ne peut contenir qu'un seul syst\`eme plan\'etaire.

Les masses sont exprimées en masse solaire.  La masse solaire de référence dépend du flag ref\_gmsun.
Les unit\'es des coordonn\'ees des plan\`etes doivent \^etre en UA, an et radians.

Sur chaque ligne, on a :
\begin{itemize}
\item colonne 1 : chaine sans espace donnant le nom du syst\`eme. Par exemple P0001 ou N0002, .... .
\item colonne 2 : nombre de plan\`etes (sans l'\'etoile) , nomm\'e nbplan.
\item colonne 3 : Masse de  l'\'etoile exprimée en masse solaire (=1 pour le système solaire)
\item colonne 4 \`a 4+nbplan-1 : Masse des plan\`etes exprimée en masse solaire 
\item colonne 4+nbplan : type de coordonn\'ees initiales des plan\`etes
\begin{itemize}
\item 1:  elliptiques h\'eliocentriques canoniques
	     CI(1:6) = (a,e,I,M,omega,Omega)
\item 2:  elliptiques h\'eliocentriques non canoniques
	     CI(1:6) = (a,e,I,M,omega,Omega)
\item 3:  elliptiques h\'eliocentriques canoniques
	     CI(1:6) = (a,la,k,h,q,p)
\item 4:  elliptiques h\'eliocentriques non canoniques
	     CI(1:6) = (a,la,k,h,q,p)
\item 5:  positions vitesses h\'eliocentriques
	     CI(1:6) = (x,y,z,vx,vy,vz)
\end{itemize}

\item colonne 4+nbplan+1 \`a  4+nbplan+6 :   coordonn\'ees initiales (6 composantes) de la plan\`ete 1
\item  colonnes suivantes :   coordonn\'ees initiales (6 composantes) pour les plan\`etes suivantes
\end{itemize}

 Par exemple, si on a 3 plan\`etes avec des positions/vitesses h\'eliocentriques, on a dans les colonnes :
 
\begin{tabular}{|c|c|c|c|c|c|c|c|c|c|} \hline
1 &  2 &  3 & 4 & 5 &6 &7 &8-13 &14-19 &20-25 \\ \hline
P0001 & 3 & $M_{star}$  & $M_1$ &  $M_2$ & $M_3$  &5 &$CI_1(1:6)$ & $CI_2(1:6)$&$CI_3(1:6)$\\    \hline
\end{tabular}

%%%%%%%%%%%%%%%%%%%%%%%%%%%%%%%%%%%%%%
\subsection{Format du fichier nf\_initpart}

Ce fichier contient les conditions initiales (coordonn\'ees) des particules. 
Ce fichier stocke une particule par ligne.

Les unit\'es des coordonn\'ees des particules doivent \^etre en UA, an et radians.

Sur chaque ligne, on a :
\begin{itemize}
\item colonne 1 : chaine sans espace donnant le nom de la particule. Par exemple P0001 ou N0002, .... .
\item colonne 2 :  type de coordonn\'ees initiales la particule.
\item colonne 3 \`a 8 :  coordonn\'ees initiales (6 composantes)  de la particule
\begin{itemize}
\item 1:  elliptiques h\'eliocentriques canoniques
	     CI(1:6) = (a,e,I,M,omega,Omega)
\item 2:  elliptiques h\'eliocentriques non canoniques
	     CI(1:6) = (a,e,I,M,omega,Omega)
\item 3:  elliptiques h\'eliocentriques canoniques
	     CI(1:6) = (a,la,k,h,q,p)
\item 4:  elliptiques h\'eliocentriques non canoniques
	     CI(1:6) = (a,la,k,h,q,p)
\item 5:  positions vitesses h\'eliocentriques
	     CI(1:6) = (x,y,z,vx,vy,vz)
\end{itemize}

\end{itemize}

Par exemple, si on a deux particules avec des positions/vitesses h\'eliocentriques, on a dans les colonnes :
 
\begin{tabular}{|c|c|c|c|c|c|c|c|c|c|} \hline
1 &  2 &  3-8 \\ \hline
P0001 &5 &$CI_1(1:6)$ \\    \hline
P0002 &5 &$CI_2(1:6)$ \\    \hline
\end{tabular}

%%%%%%%%%%%%%%%%%%%%%%%
%%%%%%%%%%%%%%%%%%%%%%%
%%%%%%%%%%%%%%%%%%%%%%%
\section{Fichiers de sortie}

%%%%%%%%%%%%%%%%%%%%%%%
\subsection{Format du fichier {\bf ???.ci\_pla} }

Ce fichier contient les conditions initiales (masses et coordonn\'ees) des syst\`emes plan\'etaires. 
Ce fichier stocke un syst\`eme plan\'etaire par ligne.

Son format est identique \`a   celui de nf\_initext.

%%%%%%%%%%%%%%%%%%%%%%%
\subsection{Format du fichier {\bf ???.ci\_part} }

Ce fichier contient les conditions initiales (coordonn\'ees) des particules. 
Ce fichier stocke une particule par ligne.

Son format est identique \`a   celui de nf\_initpart.

%%%%%%%%%%%%%%%%%%%%%%%
\subsection{Format du fichier {\bf ???.control} }

Ce fichier contient 5 colonnes et indique pour pour chaque condition initiale si l'int\'egration s'est bien d\'eroul\'ee ou non.
Ce fichier stocke un seul syst\`eme plan\'etaire.

Sur chaque ligne, on a :
\begin{itemize}
\item colonne 1 : chaine sans espace donnant le nom du syst\`eme. Par exemple P0001 ou N0002, .... .
\item colonne 2 : 
\begin{itemize}
\item 0: l'int\'egration s'est correctement termin\'ee
\item -3: probl\`eme de convergence dans kepsaut. L'int\'egration s'est interrompue.
\item -4: cas non elliptique.  L'int\'egration s'est interrompue.
\item -5: variation trop grande de l'énergie.  La colonne 6 contient la valeur absolue de l'erreur relative de l'énergie par rapport \`a l'énergie au temps 0. L'int\'egration s'est interrompue.
\item -6: corps trop proche de l'\'etoile.  La colonne 6 contient la distance de la plan\`ete \`a l'\'etoile. L'int\'egration s'est interrompue.
\item -7: corps trop loin de l'\'etoile.  La colonne 6 contient la distance de la plan\`ete \`a l'\'etoile. L'int\'egration s'est interrompue.
\end{itemize}
\item colonne 3 : temps initial de l'int\'egration
\item colonne 4 : temps finale de l'int\'egration
\item colonne 5 : corps (si disponible) ayant g\'en\'er\'e l'erreur
\item colonne 6 : 0 si aucune erreur. Sinon, elle contient une valeur d\'ependante de la colonne 2.
\end{itemize}

%%%%%%%%%%%%%%%%%%%%%%%
\subsection{Format du fichier {\bf ???.control\_part} }

Ce fichier contient 5 colonnes et indique pour pour chaque condition initiale si l'int\'egration s'est bien d\'eroul\'ee ou non.
Ce fichier stocke une particule par ligne.

Sur chaque ligne, on a :
\begin{itemize}
\item colonne 1 : chaine sans espace donnant le nom de la particule. Par exemple P0001 ou N0002, .... .
\item colonne 2 : 
\begin{itemize}
\item 0: l'int\'egration s'est correctement termin\'ee
\item -3: probl\`eme de convergence dans kepsaut. L'int\'egration s'est interrompue.
\item -4: cas non elliptique.  L'int\'egration s'est interrompue.
\item -6: corps trop proche de l'\'etoile.  La colonne 6 contient la distance de la particule \`a l'\'etoile. L'int\'egration s'est interrompue pour cette particule.
\item -7: corps trop loin de l'\'etoile.  La colonne 6 contient la distance de la particule \`a l'\'etoile. L'int\'egration s'est interrompue pour cette particule.
\item -9: corps trop proche d'une plan\`ete.  La colonne 6 contient la distance de la plan\`ete \`a l'\'etoile. L'int\'egration s'est interrompue.
\end{itemize}
\item colonne 3 : temps initial de l'int\'egration
\item colonne 4 : temps finale de l'int\'egration
\item colonne 5 : corps (si disponible) ayant g\'en\'er\'e l'erreur
\item colonne 6 : 0 si aucune erreur. Sinon, elle contient une valeur d\'ependante de la colonne 2.
\end{itemize}


%%%%%%%%%%%%%%%%%%%%%%%
\subsection{Format du fichier {\bf ???.int} }

Chaque fichier contient un seul syst\`eme plan\'etaire.
Ce fichier contient 5 colonnes et stocke la valeur des int\'egrales premi\`eres  : \'energie et moment cin\'etique.

Sur chaque ligne, on a : 

\begin{tabular}{|c|c|c|} \hline
colonne 1 &  colonne 2 & colonne 3-5 \\ \hline
temps & \'energie & moment cin\'etique (x,y,z)\\    \hline
\end{tabular}

La premi\`ere ligne contient la valeur initiale des int\'egrales premi\`eres. Les lignes suivantes contient la différence (absolue) des intégrales par rapport à la valeur initiale.


%%%%%%%%%%%%%%%%%%%%%%%
\subsection{Format du fichier {\bf ???.car} }

Ce fichier contient les positions h\'eliocentriques et vitesses h\'eliocentriques cart\'esiennes des plan\`etes. Les unit\'es sont en AU et AU/an.
Chaque fichier contient un seul syst\`eme plan\'etaire.


Sur chaque ligne, on a : 

\begin{tabular}{|c|c|c|c|} \hline
colonne 1 &   colonne 2-7 & colonne 8-13 & ... \\ \hline
temps & (x,y,z,vx,vy,vz) de la plan\`ete 1  & (x,y,z,vx,vy,vz) de la plan\`ete 2 & ... \\    \hline
\end{tabular}

%%%%%%%%%%%%%%%%%%%%%%%
\subsection{Format du fichier {\bf ???.ell} }

Ce fichier contient les \'el\'ements elliptiques des plan\`etes. le type d\'el\'ement d\'pend du param\`etres  {\bf out\_ell}. Les unit\'es sont en AU, an et radians.
Chaque fichier contient un seul syst\`eme plan\'etaire.

Sur chaque ligne, on a : 

\begin{tabular}{|c|c|c|c|} \hline
colonne 1 &   colonne 2-7 & colonne 8-13 & ... \\ \hline
temps & ell(1:6) de la plan\`ete 1  & ell(1:6) de la plan\`ete 2 & ... \\    \hline
\end{tabular}

%%%%%%%%%%%%%%%%%%%%%%%
\subsection{Format du fichier {\bf ???.car\_part} }

Ce fichier contient les positions h\'eliocentriques et vitesses h\'eliocentriques cart\'esiennes des particules. Les unit\'es sont en AU et AU/an.

 Il y a un fichier par processeur. Chaque fichier contient plusieurs particules.  Il y a une seule particule et une seule tranche de calcul  par ligne. Le fichier contient toutes les tranches d'une m\^eme  condition initiale.


Sur chaque ligne, on a : 

\begin{tabular}{|c|c|c|} \hline
colonne 1 &   colonne 2 & colonne 3-8\\ \hline
nom & temps & (x,y,z,vx,vy,vz) de la particule  \\    \hline
\end{tabular}

%%%%%%%%%%%%%%%%%%%%%%%
\subsection{Format du fichier {\bf ???.ell\_part} }

Ce fichier contient les \'el\'ements elliptiques des particules. le type d\'el\'ement d\'pend du param\`etres  {\bf out\_ell}. Les unit\'es sont en AU, an et radians.

 Il y a un fichier par processeur. Chaque fichier contient plusieurs particules.  Il y a une seule particule et une seule tranche de calcul  par ligne. Le fichier contient toutes les tranches d'une m\^eme  condition initiale.

Sur chaque ligne, on a : 

\begin{tabular}{|c|c|c|} \hline
colonne 1 &   colonne 2 & colonne 3-8\\ \hline
nom &  temps & ell(1:6) de la particule. \\    \hline
\end{tabular}

%%%%%%%%%%%%%%%%%%%%%%%
\subsection{Format du fichier {\bf ???.minmax\_aei\_part} }

Ce fichier contient les minimum, maximum et moyenne en a,e et i sur une tranche de temps. Les unit\'es sont en AU et radians.
Les types des \'el\'ements elliptiques d\'ependent du param\`etre  {\bf minmax\_aei\_elltype}.

 Il y a un fichier par processeur. Chaque fichier contient plusieurs particules.
 
 
Sur chaque ligne, on a dans chaque colonne: 

\begin{tabular}{|c|c|c|c|c|c|c|c|c|c|c|} \hline
 1 &    2 &  \multicolumn{9}{c|}{3-11}  \\ \hline
  &    &  \multicolumn{9}{c|}{particule}  \\ \hline
  &    &  \multicolumn{3}{c|}{a} & \multicolumn{3}{c|}{e} & \multicolumn{3}{c|}{i} \\ \hline
nom & temps &  min   &moy & max  &  min   & moy & max   & min   & moy & max \\  
 & final &     & &   &     &  &    &    &  &  
\\\hline
\end{tabular}

 Ici, le temps final est le temps de fin de chaque tranche. Le fichier contient toutes les tranches d'une m\^eme  condition initiale.

%%%%%%%%%%%%%%%%%%%%%%%
\subsection{Format du fichier {\bf ???.minmax\_alp\_part} }

Ce fichier contient les minimum, maximum et moyenne en $a_{part}-a_{p(1)}$, $\lambda_{part}-\lambda_{p(1)}$ et $\varpi_{part}-\varpi_{p(1)}$ sur une tranche de temps. Les unit\'es sont en AU et radians. Pour les diff\'erences d'angle, il y a une double d\'etermination entre $[0,2\pi]$ et entre $[-\pi, \pi]$.
Les types des \'el\'ements elliptiques d\'ependent du param\`etre  {\bf minmax\_alp\_elltype}.

 Il y a un fichier par processeur. Chaque fichier contient plusieurs particules.  Il y a une seule particule et une seule tranche de calcul  par ligne. Le fichier contient toutes les tranches d'une m\^eme  condition initiale.

Sur chaque ligne, on a dans chaque colonne: 

\begin{tabularx}{\textwidth}{|l|l|X|}
 \hline
 colonne &      \multicolumn{2}{c|}{description} \\ \hline
1  &    \multicolumn{2}{l|}{nom} \\ \hline
2  &    \multicolumn{2}{l|}{temps final de chaque tranche} \\ \hline
3 &    & min\\
4 &  $a_{part}-a_{p(1)}$ & moy\\
5 & &   max\\ \hline
6 & &   min\\
7 & $\lambda_{part}-\lambda_{p(1)}$ sur $[0,2\pi]$&moy\\
8 &    & max\\ \hline
9 &    & min\\
10& $\lambda_{part}-\lambda_{p(1)}$ sur $[-\pi,\pi]$ & moy\\
11 & &    max\\ \hline
12 & &    min\\
13 &   $\varpi_{part}-\varpi_{p(1)}$ sur $[0,2\pi]$ & moy\\
14 & &    max\\ \hline
15 & &    min\\
16 & $\varpi_{part}-\varpi_{p(1)}$ sur $[-\pi,\pi]$ & moy\\
17 & &    max\\ \hline
\end{tabularx}

%%%%%%%%%%%%%%%%%%%%%%%
\subsection{Format du fichier {\bf ???.naf\_alkhqp\_part} }

Ce fichier contient l'analyse en fr\'equence des particules en $a\exp^{\imath\lambda}, k+\imath h, q+\imath p$ sur une tranche de temps. Les unit\'es des fr\'equences d\'ependent de naf\_alkhqp\_dtour.

 Il y a un fichier par processeur. Chaque fichier contient plusieurs particules. Il y a une seule particule et une seule tranche de calcul  par ligne. Le fichier contient toutes les tranches d'une m\^eme  condition initiale.


Sur chaque ligne, on a dans chaque colonne: 

\begin{tabularx}{\textwidth}{|l|l|l|l|X|}
 \hline
 colonne &      \multicolumn{4}{c|}{description} \\ \hline
1  &    \multicolumn{4}{l|}{nom} \\ \hline
2  &    \multicolumn{4}{l|}{temps initial (T0) de chaque tranche} \\ \hline
3 & &   & & fr\'equence\\
4 &particule  &$a\exp^{\imath\lambda}$& terme 1 & amplitude (partie r\'eelle)\\
5 & &   & &amplitude (partie imaginaire)\\ \hline
\dots &particule  &$a\exp^{\imath\lambda}$& terme ?? &\dots \\ \hline
 & &   & &fr\'equence\\
 &particule  & $a\exp^{\imath\lambda}$ & terme $naf\_alkhqp\_nterm$ & amplitude (partie r\'eelle)\\
 & &   & &amplitude (partie imaginaire)\\ \hline
 & &   & & fr\'equence\\
 &particule  &$k+\imath h$& terme 1 & amplitude (partie r\'eelle)\\
 & &   & &amplitude (partie imaginaire)\\ \hline
\dots &particule  &$k+\imath h$& terme ?? &\dots \\ \hline
 & &   & &fr\'equence\\
 &particule  & $k+\imath j$ & terme $naf\_alkhqp\_nterm$ & amplitude (partie r\'eelle)\\
 & &   & &amplitude (partie imaginaire)\\ \hline
& &   & & fr\'equence\\
 &particule  &$q+\imath p$& terme 1 & amplitude (partie r\'eelle)\\
 & &   & &amplitude (partie imaginaire)\\ \hline
\dots &particule &$q+\imath p$& terme ?? &\dots \\ \hline
 & &   & &fr\'equence\\
 &particule & $q+\imath p$ & terme $naf\_alkhqp\_nterm$ & amplitude (partie r\'eelle)\\
 & &   & &amplitude (partie imaginaire)\\ \hline
\end{tabularx}



%%%%%%%%%%%%%%%%%%%%%%%
\subsection{Format du fichier {\bf ???.naf\_alkh\_part} }

Ce fichier contient l'analyse en fr\'equence des particules en $a\exp^{\imath\lambda}, k+\imath h$ sur une tranche de temps. Les unit\'es des fr\'equences d\'ependent de naf\_alkhqp\_dtour.

 Il y a un fichier par processeur. Chaque fichier contient plusieurs particules. Il y a une seule particule et une seule tranche de calcul  par ligne. Le fichier contient toutes les tranches d'une m\^eme  condition initiale.


Sur chaque ligne, on a dans chaque colonne: 

\begin{tabularx}{\textwidth}{|l|l|l|l|X|}
 \hline
 colonne &      \multicolumn{4}{c|}{description} \\ \hline
1  &    \multicolumn{4}{l|}{nom} \\ \hline
2  &    \multicolumn{4}{l|}{temps initial (T0) de chaque tranche} \\ \hline
3 & &   & & fr\'equence\\
4 &particule &$a\exp^{\imath\lambda}$& terme 1 & amplitude (partie r\'eelle)\\
5 & &   & &amplitude (partie imaginaire)\\ \hline
\dots & particule &$a\exp^{\imath\lambda}$& terme ?? &\dots \\ \hline
 & &   & &fr\'equence\\
 &particule & $a\exp^{\imath\lambda}$ & terme $naf\_alkhqp\_nterm$ & amplitude (partie r\'eelle)\\
 & &   & &amplitude (partie imaginaire)\\ \hline
 & &   & & fr\'equence\\
 &particule &$k+\imath h$& terme 1 & amplitude (partie r\'eelle)\\
 & &   & &amplitude (partie imaginaire)\\ \hline
\dots & particule &$k+\imath h$& terme ?? &\dots \\ \hline
 & &   & &fr\'equence\\
 &particule & $k+\imath j$ & terme $naf\_alkhqp\_nterm$ & amplitude (partie r\'eelle)\\
 & &   & &amplitude (partie imaginaire)\\ \hline
\end{tabularx}

%%%%%%%%%%%%%%%%%%%%%%%
\subsection{Format du fichier {\bf ???.naf\_diffalp\_part} }

Ce fichier contient l'analyse en fr\'equence en  $\exp^{\imath(\lambda_{part}-\lambda_{p(1)})}$ et $\exp^{\imath(\varpi_{part}-\varpi_{p(1)})}$ sur une tranche de temps. Les unit\'es des fr\'equences d\'ependent de naf\_diffalp\_dtour.

 Il y a un fichier par processeur. Chaque fichier contient plusieurs particules.  Il y a une seule particule et une seule tranche de calcul  par ligne.  Le fichier contient toutes les tranches d'une m\^eme  condition initiale.


Sur chaque ligne, on a dans chaque colonne: 

\begin{tabularx}{\textwidth}{|l|l|l|X|}
 \hline
 colonne &      \multicolumn{3}{c|}{description} \\ \hline
1  &    \multicolumn{3}{l|}{nom} \\ \hline
2  &    \multicolumn{3}{l|}{temps initial (T0) de chaque tranche} \\ \hline
3 & &    & fr\'equence\\
4 &$\exp^{\imath(\lambda_{part}-\lambda_{p(1)})}$& terme 1 & amplitude (partie r\'eelle)\\
5 &   & &amplitude (partie imaginaire)\\ \hline
\dots &  $\exp^{\imath(\lambda_{part}-\lambda_{p(1)})}$& terme ?? &\dots \\ \hline
 & &    &fr\'equence\\
 &  $\exp^{\imath(\lambda_{part}-\lambda_{p(1)})}$ & terme $naf\_alkhqp\_nterm$ & amplitude (partie r\'eelle)\\
 & &    &amplitude (partie imaginaire)\\ \hline
 & &   & fr\'equence\\
 &$\exp^{\imath(\varpi_{part}-\varpi_{p(1)})}$& terme 1 & amplitude (partie r\'eelle)\\
 &    & &amplitude (partie imaginaire)\\ \hline
\dots & $\exp^{\imath(\varpi_{part}-\varpi_{p(1)})}$& terme ?? &\dots \\ \hline
 & &    &fr\'equence\\
 & $\exp^{\imath(\varpi_{part}-\varpi_{p(1)})}$ & terme $naf\_alkhqp\_nterm$ & amplitude (partie r\'eelle)\\
 &   & &amplitude (partie imaginaire)\\ \hline
\end{tabularx}

%%%%%%%%%%%%%%%%%%%%%%%
\subsection{Format du fichier {\bf ???.naf\_alkhqp} }

Ce fichier contient l'analyse en fr\'equence des plan\`etes en $a\exp^{\imath\lambda}, k+\imath h, q+\imath p$ sur une tranche de temps. Les unit\'es des fr\'equences d\'ependent de naf\_alkhqp\_dtour.

Il y a un fichier par processeur. Chaque fichier contient un seul syst\`eme plan\'etaire. Le fichier contient toutes les tranches de ce syst\`eme plan\'etaire.


Sur chaque ligne, on a dans chaque colonne: 

\begin{tabularx}{\textwidth}{|l|l|l|l|X|}
 \hline
 colonne &      \multicolumn{4}{c|}{description} \\ \hline
1  &    \multicolumn{4}{l|}{nom} \\ \hline
2  &    \multicolumn{4}{l|}{temps initial (T0) de chaque tranche} \\ \hline
3 & &   & & fr\'equence\\
4 &planete 1 &$a\exp^{\imath\lambda}$& terme 1 & amplitude (partie r\'eelle)\\
5 & &   & &amplitude (partie imaginaire)\\ \hline
\dots & planete 1 &$a\exp^{\imath\lambda}$& terme ?? &\dots \\ \hline
 & &   & &fr\'equence\\
 &planete 1 & $a\exp^{\imath\lambda}$ & terme $naf\_alkhqp\_nterm$ & amplitude (partie r\'eelle)\\
 & &   & &amplitude (partie imaginaire)\\ \hline
 & &   & & fr\'equence\\
 &planete 1 &$k+\imath h$& terme 1 & amplitude (partie r\'eelle)\\
 & &   & &amplitude (partie imaginaire)\\ \hline
\dots & planete 1 &$k+\imath h$& terme ?? &\dots \\ \hline
 & &   & &fr\'equence\\
 &planete 1 & $k+\imath j$ & terme $naf\_alkhqp\_nterm$ & amplitude (partie r\'eelle)\\
 & &   & &amplitude (partie imaginaire)\\ \hline
& &   & & fr\'equence\\
 &planete 1 &$q+\imath p$& terme 1 & amplitude (partie r\'eelle)\\
 & &   & &amplitude (partie imaginaire)\\ \hline
\dots & planete 1 &$q+\imath p$& terme ?? &\dots \\ \hline
 & &   & &fr\'equence\\
 &planete 1 & $q+\imath p$ & terme $naf\_alkhqp\_nterm$ & amplitude (partie r\'eelle)\\
 & &   & &amplitude (partie imaginaire)\\ \hline
\dots & planete 2 &$a\exp^{\imath\lambda}$& terme 1 &fr\'equence \\ \hline
\dots & & & \dots &\\\hline
\end{tabularx}



%%%%%%%%%%%%%%%%%%%%%%%
\subsection{Format du fichier {\bf ???.naf\_alkh} }

Ce fichier contient l'analyse en fr\'equence des plan\`etes en $a\exp^{\imath\lambda}, k+\imath h$ sur une tranche de temps. Les unit\'es des fr\'equences d\'ependent de naf\_alkhqp\_dtour.

Il y a un fichier par processeur. Chaque fichier contient un seul syst\`eme plan\'etaire. Le fichier contient toutes les tranches de ce syst\`eme plan\'etaire.


Sur chaque ligne, on a dans chaque colonne: 

\begin{tabularx}{\textwidth}{|l|l|l|l|X|}
 \hline
 colonne &      \multicolumn{4}{c|}{description} \\ \hline
1  &    \multicolumn{4}{l|}{nom} \\ \hline
2  &    \multicolumn{4}{l|}{temps initial (T0) de chaque tranche} \\ \hline
3 & &   & & fr\'equence\\
4 &planete 1 &$a\exp^{\imath\lambda}$& terme 1 & amplitude (partie r\'eelle)\\
5 & &   & &amplitude (partie imaginaire)\\ \hline
\dots & planete 1 &$a\exp^{\imath\lambda}$& terme ?? &\dots \\ \hline
 & &   & &fr\'equence\\
 &planete 1 & $a\exp^{\imath\lambda}$ & terme $naf\_alkhqp\_nterm$ & amplitude (partie r\'eelle)\\
 & &   & &amplitude (partie imaginaire)\\ \hline
 & &   & & fr\'equence\\
 &planete 1 &$k+\imath h$& terme 1 & amplitude (partie r\'eelle)\\
 & &   & &amplitude (partie imaginaire)\\ \hline
\dots & planete 1 &$k+\imath h$& terme ?? &\dots \\ \hline
 & &   & &fr\'equence\\
 &planete 1 & $k+\imath j$ & terme $naf\_alkhqp\_nterm$ & amplitude (partie r\'eelle)\\
 & &   & &amplitude (partie imaginaire)\\ \hline
\dots & planete 2 &$a\exp^{\imath\lambda}$& terme 1 &fr\'equence \\ \hline
\dots & & & \dots &\\\hline
\end{tabularx}


\end{document}  