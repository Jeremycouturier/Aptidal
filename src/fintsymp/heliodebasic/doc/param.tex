\documentclass[11pt]{article}
\usepackage[utf8]{inputenc}
\usepackage[T1]{fontenc}
\usepackage{geometry}                % See geometry.pdf to learn the layout options. There are lots.
\geometry{a4paper}                   % ... or a4paper or a5paper or ... 
%\geometry{landscape}                % Activate for for rotated page geometry
%\usepackage[parfill]{parskip}    % Activate to begin paragraphs with an empty line rather than an indent
\usepackage{graphicx}
\usepackage{amssymb}
\usepackage{epstopdf}
\DeclareGraphicsRule{.tif}{png}{.png}{`convert #1 `dirname #1`/`basename #1 .tif`.png}
\usepackage{tabularx}

\title{HELIODEBASIC}
\author{Micka\"el Gastineau}

\begin{document}
\maketitle

\section{Version s\'equentielle}

\begin{itemize}
 \item Compilation:  
 
make clean

make
 
\item Execution en interactif: 

heliodebasic.x  ???.par
\item Soumission sur bessel

qsubserial -fastsse4 heliodebasic.x  ???.par
\end{itemize}


\section{Fichiers d'entree}

\subsection{Fichier de param\`etres  intplabasic.par}

\subsubsection*{Contr\^oles de l'int\'egration}

\begin{tabularx}{\textwidth}{|l|X|}
\hline
Nom du champ& Descriptif \\ \hline 
chemin   & dossier o\`u seront stock\'es les fichiers \\ \hline
 nf\_rad    & radical de tous les fichiers g\'en\'er\'es\\ \hline
 nf\_initext& fichier de conditions initiales des plan\`etes\\ \hline
 ref\_gmsun& Valeur du GM du soleil de r\'ef\'erence \\ 
& 0: valeur issue de la Table 1 de "NOMINAL VALUES FOR SELECTED SOLAR AND PLANETARY QUANTITIES: IAU 2015 RESOLUTION B3"\\
& 1: valeur calculée à partir de la constante de Gauss (k=0.01720209895e0) \\\hline


 int\_type& sch\'ema de l'int\'egrateur  \\
&'DOPRI' : utilise dopri8 \\
&'ODEX1' : utilise odex1D \\
&'ODEX2' : utilise odex2D \\
 \hline

 
 tinit & temps initial (en g\'eneral 0) \\ \hline

 dt& pas de temps de l'int\'egration en ann\'ee \\ \hline

 n\_iter& nombre de pas d'int\'egrations \`a calculer. A la fin de l'int\'egration, le temps final sera  n\_iter*dt ans.\\ \hline

 n\_out & fr\'equence d'\'ecriture des int\'egrales premi\`eres, coordonn\'ees cart\'esiennes et \'el\'ements elliptiques. Il est exprim\'e en nombre de pas d'int\'egrations. Les donn\'ees seront \'ecrites tous les n\_out*dt ann\'ees.
 \\ \hline
 out\_ell & format des \'el\'ements elliptiques \'ecrites dans les fichiers xxx.ell \\
&1:  elliptiques h\'eliocentriques canoniques\\
&	     CI(1:6) = (a,e,I,M,omega,Omega)\\
&2:  elliptiques h\'eliocentriques non canoniques\\
&	     CI(1:6) = (a,e,I,M,omega,Omega)\\
&3:  elliptiques h\'eliocentriques canoniques\\
&	     CI(1:6) = (a,la,k,h,q,p)\\
&4:  elliptiques h\'eliocentriques non canoniques\\
&	     CI(1:6) = (a,la,k,h,q,p)\\ \hline
 if\_invar & =0 , l'int\'egration se fait dans le rep\`ere actuel. \\
& =1, l'int\'egration se fait dans le plan invariant et les donn\'ees g\'en\'er\'ees sont dans ce plan invariant 
\\ \hline

 if\_int & =0 , les int\'egrales premi\`eres ne sont pas \'ecrites.\\
&=1, les int\'egrales premi\`eres sont \'ecrites dans les fichiers xxx.int. Un fichier par syst\`eme\\ \hline
 
 if\_ell & =0 , les \'el\'ements elliptiques ne sont pas \'ecrits.\\
&=1, les \'el\'ements elliptiques sont \'ecrits dans les fichiers xxx.ell. Un fichier par syst\`eme\\ \hline
 
 if\_car &  =0 , les \'el\'ements cart\'esiens (positions/vitesses) ne sont pas \'ecrits.\\
&=1, les \'el\'ements cart\'esiens positions/vitesses) sont \'ecrits dans les fichiers xxx.car. Un fichier par syst\`eme\\ \hline 
 \end{tabularx}



\subsection{Format du fichier nf\_initext}

Ce fichier contient les conditions initiales (masses et coordonn\'ees) des syst\`emes plan\'etaires. 
Ce fichier stocke un syst\`eme plan\'etaire par ligne.

Les masses sont exprimées en masse solaire.  La masse solaire de référence dépend du flag ref\_gmsun. 
Les unit\'es des coordonn\'ees des plan\`etes doivent \^etre en UA, an et radians.

Sur chaque ligne, on a :
\begin{itemize}
\item colonne 1 : chaine sans espace donnant le nom du syst\`eme. Par exemple P0001 ou N0002, .... .
\item colonne 2 : nombre de plan\`etes (sans l'\'etoile) , nomm\'e nbplan.
\item colonne 3 : Masse de  l'\'etoile exprimée en masse solaire (=1 pour le système solaire)
\item colonne 4 \`a 4+nbplan-1 :  Masse des plan\`etes exprimée en masse solaire 
\item colonne 4+nbplan : type de coordonn\'ees initiales des plan\`etes
\begin{itemize}
\item 1:  elliptiques h\'eliocentriques canoniques
	     CI(1:6) = (a,e,I,M,omega,Omega)
\item 2:  elliptiques h\'eliocentriques non canoniques
	     CI(1:6) = (a,e,I,M,omega,Omega)
\item 3:  elliptiques h\'eliocentriques canoniques
	     CI(1:6) = (a,la,k,h,q,p)
\item 4:  elliptiques h\'eliocentriques non canoniques
	     CI(1:6) = (a,la,k,h,q,p)
\item 5:  positions vitesses h\'eliocentriques
	     CI(1:6) = (x,y,z,vx,vy,vz)
\end{itemize}

\item colonne 4+nbplan+1 \`a  4+nbplan+6 :   coordonn\'ees initiales (6 composantes) de la plan\`ete 1
\item  colonnes suivantes :   coordonn\'ees initiales (6 composantes) pour les plan\`etes suivantes
\end{itemize}

 Par exemple, si on a 3 plan\`etes avec des positions/vitesses h\'eliocentriques, on a dans les colonnes :
 
\begin{tabular}{|c|c|c|c|c|c|c|c|c|c|} \hline
1 &  2 &  3 & 4 & 5 &6 &7 &8-13 &14-19 &20-25 \\ \hline
P0001 & 3 & $M_{star}$  & $M_1$ &  $M_2$ & $M_3$  &5 &$CI_1(1:6)$ & $CI_2(1:6)$&$CI_3(1:6)$\\    \hline
\end{tabular}

%%%%%%%%%%%%%%%%%%%%%%%
%%%%%%%%%%%%%%%%%%%%%%%
%%%%%%%%%%%%%%%%%%%%%%%
\section{Fichiers de sortie}

%%%%%%%%%%%%%%%%%%%%%%%
\subsection{Format du fichier {\bf ???.int} }

Chaque fichier contient un seul syst\`eme plan\'etaire.
Ce fichier contient 5 colonnes et stocke la valeur de int\'egrales premi\`eres  : \'energie du moment cin\'etique.

Sur chaque ligne, on a : 

\begin{tabular}{|c|c|c|} \hline
colonne 1 &  colonne 2 & colonne 3-5 \\ \hline
temps & \'energie & moment cin\'etique (x,y,z)\\    \hline
\end{tabular}


%%%%%%%%%%%%%%%%%%%%%%%
\subsection{Format du fichier {\bf ???.car} }

Ce fichier contient les positions h\'eliocentriques et vitesses h\'eliocentriques cart\'esiennes des plan\`etes. Les unit\'es sont en AU et AU/an.
Chaque fichier contient un seul syst\`eme plan\'etaire.


Sur chaque ligne, on a : 

\begin{tabular}{|c|c|c|c|} \hline
colonne 1 &   colonne 2-7 & colonne 8-13 & ... \\ \hline
temps & (x,y,z,vx,vy,vz) de la plan\`ete 1  & (x,y,z,vx,vy,vz) de la plan\`ete 2 & ... \\    \hline
\end{tabular}

%%%%%%%%%%%%%%%%%%%%%%%
\subsection{Format du fichier {\bf ???.ell} }

Ce fichier contient les \'el\'ements elliptiques des plan\`etes. le type d\'el\'ement d\'pend du param\`etres  {\bf out\_ell}. Les unit\'es sont en AU, an et radians.
Chaque fichier contient un seul syst\`eme plan\'etaire.

Sur chaque ligne, on a : 

\begin{tabular}{|c|c|c|c|} \hline
colonne 1 &   colonne 2-7 & colonne 8-13 & ... \\ \hline
temps & ell(1:6) de la plan\`ete 1  & ell(1:6) de la plan\`ete 2 & ... \\    \hline
\end{tabular}


\end{document}  