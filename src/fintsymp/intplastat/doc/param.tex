\documentclass[11pt]{article}
\usepackage[utf8]{inputenc}
\usepackage[T1]{fontenc}
\usepackage{geometry}                % See geometry.pdf to learn the layout options. There are lots.
\geometry{a4paper}                   % ... or a4paper or a5paper or ... 
%\geometry{landscape}                % Activate for for rotated page geometry
%\usepackage[parfill]{parskip}    % Activate to begin paragraphs with an empty line rather than an indent
\usepackage{graphicx}
\usepackage{amssymb}
\usepackage{epstopdf}
\DeclareGraphicsRule{.tif}{png}{.png}{`convert #1 `dirname #1`/`basename #1 .tif`.png}
\usepackage{tabularx}
\usepackage{hyperref}  

\title{INTPLASTAT}
\author{Micka\"el Gastineau}

\begin{document}
\maketitle

\section{Version s\'equentielle}

\begin{itemize}
 \item Compilation:  
 
make clean
make
 
\item Execution en interactif: 

intplastat.x  ???.par
\item Soumission sur bessel

qsubserial -fastsse4 intplastat.x  ???.par
\end{itemize}

\section{Version MPI}

\begin{itemize}
 \item Compilation:  
 
 make mpi
 
\item Soumission sur bessel (ici 48 coeurs) : 

qsubmpi 48 -fastsse4 -outdir STDIN/ intplastat\_mpi.x  ???.par


\item Fusion des fichiers mpi des diff\'rents processeurs (ici, chemin="DATA" et nf\_rad="sim2014XX" )

mergempi.sh DATA sim2014XX

\end{itemize}

Les conditions initiales sont envoy\'ees une par une par le maitre aux esclaves au fil de l'eau et le maitre ne fait rien d'autre.


\section{Fichiers d'entree}

\subsection{Fichier de param\`etres  intplastat.par}

\subsubsection*{Contr\^oles de l'int\'egration}

\begin{tabularx}{\textwidth}{|l|X|}
\hline
Nom du champ& Descriptif \\ \hline 
chemin   & dossier o\`u seront stock\'es les fichiers \\ \hline
 nf\_rad    & radical de tous les fichiers g\'en\'er\'es\\ \hline
 nf\_initext& fichier de conditions initiales des plan\`etes\\
 &si if\_dump=1, fichier de red\'emarrage xxx.dump \\ \hline

 ref\_gmsun& Valeur du GM du soleil de r\'ef\'erence \\ 
& 0: valeur issue de la Table 1 de "NOMINAL VALUES FOR SELECTED SOLAR AND PLANETARY QUANTITIES: IAU 2015 RESOLUTION B3"\\
& 1: valeur calculée à partir de la constante de Gauss (k=0.01720209895e0) \\\hline

 int\_type& sch\'ema de l'int\'egrateur (e.g., 'ABA4' ou 'ABAH4' (liste \ref{autresschemas}))) \\ \hline

 type\_pas & =0 , int\'egrateur \`a pas fixe.\\
&=1, int\'egrateur \`a pas variable (cf. A. petit , 2019).\\ \hline
 
 tinit & temps initial (en g\'eneral 0)\\
  &si if\_dump=1, temps auquel on prend les conditions initiales dans le fichier xxx.dump \\ \hline

 dt& pas de temps de l'int\'egration en ann\'ee \\ \hline

 n\_iter& nombre de pas d'int\'egrations \`a calculer. A la fin de l'int\'egration, le temps final sera  n\_iter*dt ans.\\ \hline

 n\_out & fr\'equence d'\'ecriture des int\'egrales premi\`eres, coordonn\'ees cart\'esiennes et \'el\'ements elliptiques. Il est exprim\'e en nombre de pas d'int\'egrations. Les donn\'ees seront \'ecrites tous les n\_out*dt ann\'ees.
 \\ \hline
 out\_ell & format des \'el\'ements elliptiques \'ecrites dans les fichiers xxx.ell \\
&1:  elliptiques h\'eliocentriques canoniques\\
&	     CI(1:6) = (a,e,I,M,omega,Omega)\\
&2:  elliptiques h\'eliocentriques non canoniques\\
&	     CI(1:6) = (a,e,I,M,omega,Omega)\\
&3:  elliptiques h\'eliocentriques canoniques\\
&	     CI(1:6) = (a,la,k,h,q,p)\\
&4:  elliptiques h\'eliocentriques non canoniques\\
&	     CI(1:6) = (a,la,k,h,q,p)\\ \hline
 n\_dump & fr\'equence d'\'ecriture des fichiers de red\'emarrage xxx.dump. Il est exprim\'e en nombre de pas d'int\'egrations. Les donn\'ees seront \'ecrites tous les n\_dump*dt ann\'ees.
 \\ \hline
 if\_invar & =0 , l'int\'egration se fait dans le rep\`ere actuel. \\
& =1, l'int\'egration se fait dans le plan invariant et les donn\'ees g\'en\'er\'ees sont dans ce plan invariant 
\\ \hline

 if\_int & =0 , les int\'egrales premi\`eres ne sont pas \'ecrites.\\
&=1, les int\'egrales premi\`eres sont \'ecrites dans les fichiers xxx.int. Un fichier par syst\`eme\\ \hline
 
 if\_ell & =0 , les \'el\'ements elliptiques ne sont pas \'ecrits.\\
&=1, les \'el\'ements elliptiques sont \'ecrits dans les fichiers xxx.ell. Un fichier par syst\`eme\\ \hline
 
 if\_car &  =0 , les \'el\'ements cart\'esiens (positions/vitesses) ne sont pas \'ecrits.\\
&=1, les \'el\'ements cart\'esiens positions/vitesses) sont \'ecrits dans les fichiers xxx.car. Un fichier par syst\`eme\\ \hline 
 \end{tabularx}
\begin{tabularx}{\textwidth}{|l|X|}
	\hline
	Nom du champ& Descriptif \\ \hline 
 if\_dump &  =0 , les fichiers de red\'emarrage ne sont pas \'ecrits.\\
&=1, les donn\'ees pour un red\'emarrage sont \'ecrits dans les fichiers xxx.dump. Un fichier par syst\`eme\\ \hline 
 \end{tabularx}


%%%%%%%%%%%%%%%%%%%%%%%%%%%%%%%%%%%%%%
\vspace{0.5cm}
\subsubsection*{Calcul des minimum, moyenne et maximum en a,e,I}
Cela g\'en\`ere les fichiers xxx.minmax\_aei.

\begin{tabularx}{\textwidth}{|l|X|}
\hline
Nom du champ& Descriptif \\ \hline \hline
minmax\_aei\_compute &  =0, les minimum, moyenne et maximum en a,e,I ne sont pas calcul\'ees. Tous les autres champs sont ignor\'es.\\
&=1, les minimum, moyenne et maximum en a,e,I sont calcul\'es. Un fichier par processeur.\\ \hline

minmax\_aei\_stepcalc  & fr\'equence de calcul des minimum, moyenne et maximum en a,e,I. Il est exprim\'e en nombre de pas d'int\'egrations. Les donn\'ees seront calcul\'ees tous les minmax\_aei\_stepcalc*dt ann\'ees. \\ \hline

minmax\_aei\_stepout  & Longueur en nombre d'it\'erations sur laquelle on effectue les calculs de minimum, moyenne et maximum en a,e,I. Les minimum, moyenne et maximum en a,e,I sont \'ecrites tous les minmax\_aei\_stepout*dt ann\'ees dans les fichiers minmax\_aei.\\ \hline

minmax\_aei\_elltype  & Type des \'el\'ements elliptiques utilis\'e pour le calcul des  minimum, moyenne et maximum en a,e,I\\ 
&1:  elliptiques h\'eliocentriques canoniques\\
&	     CI(1:6) = (a,e,I,M,omega,Omega)\\
&2:  elliptiques h\'eliocentriques non canoniques\\
&	     CI(1:6) = (a,e,I,M,omega,Omega)\\ \hline

 \end{tabularx}

%%%%%%%%%%%%%%%%%%%%%%%%%%%%%%%%%%%%%%
\vspace{0.5cm}
\subsubsection*{Calcul des minimum, moyenne et maximum en diff\'erence d'\'el\'ements elliptiques : $a_{p(1)}-a_{p(2)}$, $\lambda_{p(1)}-\lambda_{p(2)}$ et $\varpi_{p(1)}-\varpi_{p(2)}$ avec uen double détermination des angles}
Cela g\'en\`ere les fichiers xxx.minmax\_alp.
Une double détermination est réalisée pour les angles entre $[-\pi,\pi]$ et $[0,2\pi]$.

\begin{tabularx}{\textwidth}{|l|X|}
\hline
minmax\_diffalp\_compute  & =0, les minimum, moyenne et maximum en  $a_{p(1)}-a_{p(2)}$, $\lambda_{p(1)}-\lambda_{p(2)}$ et $\varpi_{p(1)}-\varpi_{p(2)}$ ne sont pas calcul\'ees. Tous les autres champs sont ignor\'es.\\
&=1, les minimum, moyenne et maximum en  $a_{p(1)}-a_{p(2)}$, $\lambda_{p(1)}-\lambda_{p(2)}$ et $\varpi_{p(1)}-\varpi_{p(2)}$ sont calcul\'es. Un fichier par processeur.\\ \hline
minmax\_diffalp\_stepcalc   & fr\'equence de calcul des minimum, moyenne et maximum en a, $\lambda$ et $\varpi$. Il est exprim\'e en nombre de pas d'int\'egrations. Les donn\'ees seront calcul\'ees tous les minmax\_diffalp\_stepcalc*dt ann\'ees.\\ \hline

minmax\_diffalp\_stepout   &Longueur en nombre d'it\'erations sur laquelle on effectue les calculs de minimum, moyenne et maximum en a, $\lambda$ et $\varpi$. Les minimum, moyenne et maximum en a, $\lambda$ et $\varpi$ sont \'ecrites tous les minmax\_diffalp\_stepout*dt ann\'ees dans les fichiers minmax\_alp. \\ \hline
minmax\_diffalp\_elltype  & Type des \'el\'ements elliptiques utilis\'e pour le calcul des  minimum, moyenne et maximum en $a_{p(1)}-a_{p(2)}$, $\lambda_{p(1)}-\lambda_{p(2)}$ et $\varpi_{p(1)}-\varpi_{p(2)}$ \\
&6:  elliptiques h\'eliocentriques non canoniques\\
&	     CI(1:6) = (a,e,I,$\lambda$,$\varpi$,Omega)\\ \hline
minmax\_diffalp\_pla(1)   & indice de la premi\`ere plan\`ete $p(1)$. Les indices commencent \`a 1.\\ \hline
minmax\_diffalp\_pla(2)   & indice de la deuxi\`eme plan\`ete $p(2)$. Les indices commencent \`a  1.\\ \hline
 \end{tabularx}

%%%%%%%%%%%%%%%%%%%%%%%%%%%%%%%%%%%%%%
\vspace{0.5cm}
\subsubsection*{Calcul des minimum, moyenne et maximum en diff\'erence d'\'el\'ements elliptiques : $a_{p(1)}-a_{p(2)}$, $\lambda_{p(1)}-\lambda_{p(2)}$ et $\varpi_{p(1)}-\varpi_{p(2)}$ avec des angles redressés}
Cela g\'en\`ere les fichiers xxx.minmax\_alc.
Les angles sont redressés (pour être continu) avant d'effectuer les calculs. 

\begin{tabularx}{\textwidth}{|l|X|}
\hline
minmax\_diffalc\_compute  & =0, les minimum, moyenne et maximum en  $a_{p(1)}-a_{p(2)}$, $\lambda_{p(1)}-\lambda_{p(2)}$ et $\varpi_{p(1)}-\varpi_{p(2)}$ ne sont pas calcul\'ees. Tous les autres champs sont ignor\'es.\\
&=1, les minimum, moyenne et maximum en  $a_{p(1)}-a_{p(2)}$, $\lambda_{p(1)}-\lambda_{p(2)}$ et $\varpi_{p(1)}-\varpi_{p(2)}$ sont calcul\'es. Un fichier par processeur.\\ \hline
minmax\_diffalc\_stepcalc   & fr\'equence de calcul des minimum, moyenne et maximum en a, $\lambda$ et $\varpi$. Il est exprim\'e en nombre de pas d'int\'egrations. Les donn\'ees seront calcul\'ees tous les minmax\_diffalc\_stepcalc*dt ann\'ees.\\ \hline

minmax\_diffalc\_stepout   &Longueur en nombre d'it\'erations sur laquelle on effectue les calculs de minimum, moyenne et maximum en a, $\lambda$ et $\varpi$. Les minimum, moyenne et maximum en a, $\lambda$ et $\varpi$ sont \'ecrites tous les minmax\_diffalp\_stepout*dt ann\'ees dans les fichiers minmax\_alp. \\ \hline
minmax\_diffalc\_elltype  & Type des \'el\'ements elliptiques utilis\'e pour le calcul des  minimum, moyenne et maximum en $a_{p(1)}-a_{p(2)}$, $\lambda_{p(1)}-\lambda_{p(2)}$ et $\varpi_{p(1)}-\varpi_{p(2)}$ \\
&6:  elliptiques h\'eliocentriques non canoniques\\
&	     CI(1:6) = (a,e,I,$\lambda$,$\varpi$,Omega)\\ \hline
minmax\_diffalc\_pla(1)   & indice de la premi\`ere plan\`ete $p(1)$. Les indices commencent \`a 1.\\ \hline
minmax\_diffalc\_pla(2)   & indice de la deuxi\`eme plan\`ete $p(2)$. Les indices commencent \`a  1.\\ \hline
 \end{tabularx}

%%%%%%%%%%%%%%%%%%%%%%%%%%%%%%%%%%%%%%
\vspace{0.5cm}
\subsubsection*{Calcul des minimum, moyenne et maximum en diff\'erence d'\'el\'ements elliptiques : $(a_{p(1)}-a_{p(2)})^2$, $(e_{p(1)}-e_{p(2)})^2$ et $(a_{p(1)}-a_{p(2)})^2+(e_{p(1)}-e_{p(2)})^2$}
Cela g\'en\`ere les fichiers xxx.minmax\_ae2.

\begin{tabularx}{\textwidth}{|l|X|}
\hline
minmax\_diffae2\_compute  & =0, les minimum, moyenne et maximum en $(a_{p(1)}-a_{p(2)})^2$, $(e_{p(1)}-e_{p(2)})^2$ et $(a_{p(1)}-a_{p(2)})^2+(e_{p(1)}-e_{p(2)})^2$ ne sont pas calcul\'ees. Tous les autres champs sont ignor\'es.\\
&=1, les minimum, moyenne et maximum en  $(a_{p(1)}-a_{p(2)})^2$, $(e_{p(1)}-e_{p(2)})^2$ et $(a_{p(1)}-a_{p(2)})^2+(e_{p(1)}-e_{p(2)})^2$ sont calcul\'es. Un fichier par processeur.\\ \hline
minmax\_diffae2\_stepcalc   & fr\'equence de calcul des minimum, moyenne et maximum en a et e. Il est exprim\'e en nombre de pas d'int\'egrations. Les donn\'ees seront calcul\'ees tous les minmax\_diffae2\_stepcalc*dt ann\'ees.\\ \hline

minmax\_diffae2\_stepout   &Longueur en nombre d'it\'erations sur laquelle on effectue les calculs de minimum, moyenne et maximum en a et e. Les minimum, moyenne et maximum en a et e sont \'ecrites tous les minmax\_diffae2\_stepout*dt ann\'ees dans les fichiers minmax\_ae2. \\ \hline
minmax\_diffae2\_elltype  & Type des \'el\'ements elliptiques utilis\'e pour le calcul des  minimum, moyenne et maximum en $(a_{p(1)}-a_{p(2)})^2$, $(e_{p(1)}-e_{p(2)})^2$ et $(a_{p(1)}-a_{p(2)})^2+(e_{p(1)}-e_{p(2)})^2$ \\
&6:  elliptiques h\'eliocentriques non canoniques\\
&	     CI(1:6) = (a,e,I,$\lambda$,$\varpi$,Omega)\\ \hline
minmax\_diffae2\_pla(1)   & indice de la premi\`ere plan\`ete $p(1)$. Les indices commencent \`a 1.\\ \hline
minmax\_diffae2\_pla(2)   & indice de la deuxi\`eme plan\`ete $p(2)$. Les indices commencent \`a  1.\\ \hline
 \end{tabularx}

%%%%%%%%%%%%%%%%%%%%%%%%%%%%%%%%%%%%%%
\vspace{0.5cm}
\subsubsection*{Analyse en fr\'equence en $a\exp^{\imath\lambda}, k+\imath h, q+\imath p$}
Cela g\'en\`ere les fichiers xxx.naf\_alkhqp ou xxx.naf\_alkh selon la variable naf\_alkhqp\_compute.

\begin{tabularx}{\textwidth}{|l|X|}
\hline
Nom du champ& Descriptif \\ \hline \hline
naf\_alkhqp\_compute &  =0, l'analyse en fr\'equence en $a\exp^{\imath\lambda}, k+\imath h, q+\imath p$ n'est pas calcul\'ee. Tous les autres champs sont ignor\'es.\\
&=1, l'analyse en fr\'equence en $a\exp^{\imath\lambda}, k+\imath h, q+\imath p$ est calcul\'e. Un fichier par processeur avec l'extension naf\_alkhqp.\\ 
&=2, l'analyse en fr\'equence en $a\exp^{\imath\lambda}, k+\imath h$ est calcul\'e (utile pour le cas  plan (q=p=0)). Un fichier par processeur avec l'extension naf\_alkh.\\ \hline

naf\_alkhqp\_stepcalc  & fr\'equence des points utilis\'es pour l'analyse en fr\'equence. Il est exprim\'e en nombre de pas d'int\'egrations. Les entr\'ees de l'analyse en fr\'equence seront calcul\'ees tous les naf\_alkhqp\_stepcalc*dt ann\'ees. \\ \hline

naf\_alkhqp\_stepout  & Longueur en nombre d'it\'erations sur laquelle on effectue l'analyse en fr\'equence. Le r\'esultat de l'analyse en fr\'equence est \'ecrit tous les naf\_alkhqp\_stepout*dt ann\'ees dans les fichiers naf\_alkhqp ou naf\_alkh.\\ \hline

naf\_alkhqp\_elltype  & Type des \'el\'ements elliptiques utilis\'e pour le calcul de l'analyse en fr\'equence\\ 
&3:  elliptiques h\'eliocentriques canoniques\\
&	     CI(1:6) = (a,la,k,h,q,p)\\
&4:  elliptiques h\'eliocentriques non canoniques\\
&	     CI(1:6) = (a,la,k,h,q,p)\\ \hline
naf\_alkhqp\_nterm  & Nombre de termes recherch\'es pour l'analyse en fr\'equence.\\ \hline
naf\_alkhqp\_isec  & =0, la m\'ethode des secantes n'est pas utilis\'ee.\\
&=1, la m\'ethode des secantes est utilis\'ee.\\ \hline
naf\_alkhqp\_iw  & pr\'esence de fen\^etre.\\
&=-1, fenetre exponentielle PHI(T) = $1/CE*EXP(-1/(1-T^2))$ avec CE= 0.22199690808403971891E0\\
&=0, pas de fen\^etre.\\
&= $N>0$ : PHI(T) = CN*(1+COS(PI*T))**N avec CN = $2^N(N!)^2/(2N)!$\\ \hline
naf\_alkhqp\_dtour  & Longueur d'un tour de cadran ( en g\'en\'eral $2\pi$).\\ \hline
naf\_alkhqp\_tol  &  Tol\'erance pour d\'eterminer si deux fr\'equences sont identiques.\\ \hline

 \end{tabularx}


%%%%%%%%%%%%%%%%%%%%%%%%%%%%%%%%%%%%%%
\vspace{0.5cm}
\subsubsection*{Analyse en fr\'equence en $\exp^{\imath(\lambda_{p(1)}-\lambda_{p(2)})}$ et $\exp^{\imath(\varpi_{p(1)}-\varpi_{p(2)})}$}
Cela g\'en\`ere les fichiers xxx.naf\_diffalp.

\begin{tabularx}{\textwidth}{|l|X|}
\hline
Nom du champ& Descriptif \\ \hline \hline
naf\_diffalp\_compute &=0, l'analyse en fr\'equence en $\exp^{\imath(\lambda_{p(1)}-\lambda_{p(2)})}$ et $\exp^{\imath(\varpi_{p(1)}-\varpi_{p(2)})}$ n'est pas calcul\'ee. Tous les autres champs sont ignor\'es.\\
&=1, l'analyse en fr\'equence en  $\exp^{\imath(\lambda_{p(1)}-\lambda_{p(2)})}$ et $\exp^{\imath(\varpi_{p(1)}-\varpi_{p(2)})}$ est calcul\'e. Un fichier par processeur avec l'extension naf\_diffalp.\\ \hline

naf\_diffalp\_stepcalc  & fr\'equence des points utilis\'es pour l'analyse en fr\'equence. Il est exprim\'e en nombre de pas d'int\'egrations. Les entr\'ees de l'analyse en fr\'equence seront calcul\'ees tous les naf\_diffalp\_stepcalc*dt ann\'ees. \\ \hline

naf\_diffalp\_stepout  & Longueur en nombre d'it\'erations sur laquelle on effectue l'analyse en fr\'equence. Le r\'esultat de l'analyse en fr\'equence est \'ecrit tous les naf\_diffalp\_stepout*dt ann\'ees dans les fichiers naf\_diffalp ou naf\_alkh.\\ \hline

naf\_diffalp\_elltype  & Type des \'el\'ements elliptiques utilis\'es pour le calcul de l'analyse en fr\'equence\\ 
&6:  elliptiques h\'eliocentriques non canoniques\\
&	     CI(1:6) = (a,e,I,$\lambda$,$\varpi$,Omega)\\ \hline
naf\_diffalp\_nterm  & Nombre de termes recherch\'es pour l'analyse en fr\'equence.\\ \hline
naf\_diffalp\_isec  & =0, la m\'ethode des secantes n'est pas utilis\'ee.\\
&=1, la m\'ethode des secantes est utilis\'ee.\\ \hline
naf\_diffalp\_iw  & pr\'esence de fen\^etre.\\
&=-1, fenetre exponentielle PHI(T) = $1/CE*EXP(-1/(1-T^2))$ avec CE= 0.22199690808403971891E0\\
&=0, pas de fen\^etre.\\
&= $N>0$ : PHI(T) = CN*(1+COS(PI*T))**N avec CN = $2^N(N!)^2/(2N)!$\\ \hline
naf\_diffalp\_dtour  & Longueur d'un tour de cadran ( en g\'en\'eral $2\pi$).\\ \hline
naf\_diffalp\_tol  &  Tol\'erance pour d\'eterminer si deux fr\'equences sont identiques.\\ \hline
naf\_diffalp\_pla(1)   & indice de la premi\`ere plan\`ete $p(1)$. Les indices commencent \`a 1.\\ \hline
naf\_diffalp\_pla(2)   & indice de la deuxi\`eme plan\`ete $p(2)$. Les indices commencent \`a  1.\\ \hline

 \end{tabularx}

%%%%%%%%%%%%%%%%%%%%%%%%%%%%%%%%%%%%%%
\vspace{0.5cm}
\subsubsection*{Analyse en fr\'equence en $(\lambda_{p(1)}-\lambda_{p(2)})$ et $(\varpi_{p(1)}-\varpi_{p(2)})$}
Cela g\'en\`ere les fichiers xxx.naf\_difflpm.

Les analyses en fr\'equence sont effectu\'ees sur $[0,2\pi]$ et sur $[-\pi,\pi]$.

\begin{tabularx}{\textwidth}{|l|X|}
\hline
Nom du champ& Descriptif \\ \hline \hline
naf\_difflpm\_compute &=0, l'analyse en fr\'equence en $(\lambda_{p(1)}-\lambda_{p(2)})+0\imath$ et $(\varpi_{p(1)}-\varpi_{p(2)})+0\imath$ n'est pas calcul\'ee. Tous les autres champs sont ignor\'es.\\
&=1, l'analyse en fr\'equence en  $(\lambda_{p(1)}-\lambda_{p(2)})+0\imath$ et $(\varpi_{p(1)}-\varpi_{p(2)})+0\imath$ est calcul\'e. Un fichier par processeur avec l'extension naf\_difflpm.\\ \hline

naf\_difflpm\_stepcalc  & fr\'equence des points utilis\'es pour l'analyse en fr\'equence. Il est exprim\'e en nombre de pas d'int\'egrations. Les entr\'ees de l'analyse en fr\'equence seront calcul\'ees tous les naf\_difflpm\_stepcalc*dt ann\'ees. \\ \hline

naf\_difflpm\_stepout  & Longueur en nombre d'it\'erations sur laquelle on effectue l'analyse en fr\'equence. Le r\'esultat de l'analyse en fr\'equence est \'ecrit tous les naf\_difflpm\_stepout*dt ann\'ees dans les fichiers naf\_difflpm ou naf\_alkh.\\ \hline

naf\_difflpm\_elltype  & Type des \'el\'ements elliptiques utilis\'es pour le calcul de l'analyse en fr\'equence\\ 
&6:  elliptiques h\'eliocentriques non canoniques\\
&	     CI(1:6) = (a,e,I,$\lambda$,$\varpi$,Omega)\\ \hline
naf\_difflpm\_nterm  & Nombre de termes recherch\'es pour l'analyse en fr\'equence.\\ \hline
naf\_difflpm\_isec  & =0, la m\'ethode des secantes n'est pas utilis\'ee.\\
&=1, la m\'ethode des secantes est utilis\'ee.\\ \hline
naf\_difflpm\_iw  & pr\'esence de fen\^etre.\\
&=-1, fenetre exponentielle PHI(T) = $1/CE*EXP(-1/(1-T^2))$ avec CE= 0.22199690808403971891E0\\
&=0, pas de fen\^etre.\\
&= $N>0$ : PHI(T) = CN*(1+COS(PI*T))**N avec CN = $2^N(N!)^2/(2N)!$\\ \hline
naf\_difflpm\_dtour  & Longueur d'un tour de cadran ( en g\'en\'eral $2\pi$).\\ \hline
naf\_difflpm\_tol  &  Tol\'erance pour d\'eterminer si deux fr\'equences sont identiques.\\ \hline
naf\_difflpm\_pla(1)   & indice de la premi\`ere plan\`ete $p(1)$. Les indices commencent \`a 1.\\ \hline
naf\_difflpm\_pla(2)   & indice de la deuxi\`eme plan\`ete $p(2)$. Les indices commencent \`a  1.\\ \hline

 \end{tabularx}

%%%%%%%%%%%%%%%%%%%%%%%%%%%%%%%%%%%%%%
\vspace{0.5cm}
\subsubsection*{Contr\^ole de l'\'energie}
Cela arr\^ete l'int\'egration si l'erreur relative sur l'\'energie varie trop. La valeur de l'\'energie au temps initial est pris comme r\'ef\'erence.

\begin{tabularx}{\textwidth}{|l|X|}
\hline
Nom du champ& Descriptif \\ \hline \hline
ctrl\_energie\_compute & =0, le contr\^ole de l'\'energie n'est pas r\'ealis\'e. Tous les autres champs sont ignor\'es.\\
&=1, le contr\^ole de l'\'energie est r\'ealis\'e.\\  \hline
ctrl\_energie\_stepcalc & fr\'equence des points de  contr\^ole de l'\'energie. Il est exprim\'e en nombre de pas d'int\'egrations. La variation de l'\'energie sera v\'erifi\'ee tous les ctrl\_energie\_stepcalc*dt ann\'ees. \\ \hline
ctrl\_energie\_relenermax& valeur maximale de la variation de l'erreur relative sur l'\'energie. Si l'erreur relative d\'epasse cette valeur, l'int\'egration s'arr\^ete. La valeur de l'\'energie au temps initial est pris comme r\'ef\'erence.\\ \hline
 \end{tabularx}

%%%%%%%%%%%%%%%%%%%%%%%%%%%%%%%%%%%%%%
\vspace{0.5cm}
\subsubsection*{Contr\^ole de la distance \`a l'\'etoile}
Cela arr\^ete l'int\'egration si une plan\`ete s'approche trop pr\`es ou s'\'eloigne trop de l'\'etoile. 

Si ctrl\_diststar\_compute =2, cela g\'en\`ere un fichier xxx.ctrlstar\_car.

Si ctrl\_diststar\_compute =3, cela g\'en\`ere un fichier xxx.ctrlstar\_ell. 

\begin{tabularx}{\textwidth}{|l|X|}
\hline
Nom du champ& Descriptif \\ \hline \hline
ctrl\_diststar\_compute & =0, le contr\^ole de distance n'est pas r\'ealis\'e. Tous les autres champs sont ignor\'es.\\
&=1, le contr\^ole de distance est r\'ealis\'e et aucun dump n'est r\'ealis\'e.\\  
&=2, le contr\^ole de distance est r\'ealis\'e et un dump en coordon\'ees cart\'esiennes est effectu\'e.\\ 
&=3, le contr\^ole de distance est r\'ealis\'e et un dump en coordon\'ees elliptiques est effectu\'e.\\  \hline
ctrl\_diststar\_stepcalc & fr\'equence des points de  contr\^ole de distance. Il est exprim\'e en nombre de pas d'int\'egrations. La distance sera v\'erifi\'ee tous les ctrl\_diststar\_stepcalc*dt ann\'ees. \\ \hline
ctrl\_diststar\_distmin & distance minimale en UA \`a l'\'etoile. Si une plan\`ete a une distance \`a l'\'etoile inf\'erieure \`a cette valeur, l'int\'egration s'arr\^ete. \\ \hline
ctrl\_diststar\_distmax & distance maximale en UA \`a l'\'etoile. Si une plan\`ete a une distance \`a l'\'etoile sup\'erieure \`a cette valeur, l'int\'egration s'arr\^ete. \\ \hline
 \end{tabularx}

 %%%%%%%%%%%%%%%%%%%%%%%%%%%%%%%%%%%%%%
\vspace{0.5cm}
\subsubsection*{Contr\^ole de la distance entres plan\`etes}
Cela arr\^ete l'int\'egration si une plan\`ete s'approche d'une autre plan\`ete. On définit une "boule interdite" (bas\'e sur un rayon) autour de chaque plan\`ete dans le fichier ctrl\_distpla\_nfdistmin. Chaque plan\`ete a son propre "rayon" de boule interdite.
Donc d\`es qu'une autre plan\`ete est \`a l'intérieure de cette boule, on arr\^ete l'int\'egration.  

Si ctrl\_distpla\_compute =2, cela g\'en\`ere un fichier xxx.ctrlpla\_car.

Si ctrl\_distpla\_compute =3, cela g\'en\`ere un fichier xxx.ctrlpla\_ell. 

\begin{tabularx}{\textwidth}{|l|X|}
\hline
Nom du champ& Descriptif \\ \hline \hline
ctrl\_distpla\_compute & =0, le contr\^ole de distance n'est pas r\'ealis\'e. Tous les autres champs sont ignor\'es.\\
&=1, le contr\^ole de distance est r\'ealis\'e et aucun dump n'est r\'ealis\'e.\\  
&=2, le contr\^ole de distance est r\'ealis\'e et un dump en coordon\'ees cart\'esiennes est effectu\'e.\\
&=3, le contr\^ole de distance est r\'ealis\'e et un dump en coordon\'ees elliptiques est effectu\'e.\\  \hline
ctrl\_distpla\_stepcalc & fr\'equence des points de  contr\^ole de distance. Il est exprim\'e en nombre de pas d'int\'egrations. La distance sera v\'erifi\'ee tous les |ctrl\_distpla\_stepcalc|*dt ann\'ees. \\
& si ce nombre est positif, le contrôle s'effectue sur les pas de sortie. \\
& si ce nombre est n\'egatif, le contrôle s'effectue sur les pas du sch\'ema symplectique. \\ \hline
ctrl\_distpla\_nfdistmin & Nom du fichier contenant les distances minimales entre les plan\`etes. \\ \hline
 \end{tabularx}

 %%%%%%%%%%%%%%%%%%%%%%%%%%%%%%%%%%%%%%
\subsection{Sch\'ema d'intégration disponibles}\label{autresschemas}



\begin{tabularx}{\textwidth}{|l|X|}
\hline
&Variables h\'eliocentriques\\
\hline
ABAH4 & \\
ABAH5 & \\
ABAH6 & \\
ABAH7 & \\
ABAH8 & \\
ABAH9 & \\
ABAH10 & \\
ABA82 & Laskar $SABA_4$ and McLahan (8,2) \\
ABA82 & McLahan (8,4) \\
ABA844 & Blanes (8,4,4) \\
ABAH864  & Blanes (8,6,4) \\
ABAH1064 & Blanes (10,6,4) \\
&\\
BABH4& \\
BABH5& \\
BABH6& \\
BABH7& \\
BABH8& \\
BABH9& \\
BABH10& \\
BABH82 &  Laskar $SBAB_4$ and McLahan (8,2) \\
BABH84 &  McLahan (8,4) \\
BABH844 & Blanes (8,4,4) \\
BABH864  & Blanes (8,6,4) \\
BABH1064 & Blanes (10,6,4) \\
\hline
\end{tabularx}

"High order symplectic integrators for perturbed Hamiltonian systems".
J. Laskar, P. Robutel, 2010


"New families of symplectic splitting methods for numerical integration in dynamical astronomy". Blanes, Casas, Farres, Laskar, Makazaga, Murua, 2013 

\begin{tabularx}{\textwidth}{|l|X|}
\hline
&Variables de Jacobi\\
\hline
ABA4& \\
ABA5& \\
ABA6& \\
ABA7& \\
ABA8& \\
ABA9& \\
ABA10& \\
ABA82 & Laskar $SABA_4$ and McLahan (8,2) \\
ABA864  & Blanes (8,6,4) \\
ABA1064 & Blanes (10,6,4) \\
ABA104 & Blanes (10,4) \\
& \\
BAB4& \\
BAB5& \\
BAB6& \\
BAB7& \\
BAB8& \\
BAB9& \\
BAB10& \\
BAB82 &  Laskar $SBAB_4$ and McLahan (8,2) \\
BAB84 &  McLahan (8,4) \\
BAB864  & Blanes (8,6,4) \\
\hline
\end{tabularx}

"High order symplectic integrators for perturbed Hamiltonian systems".
J. Laskar, P. Robutel, 2010

"New families of symplectic splitting methods for numerical integration in dynamical astronomy". Blanes, Casas, Farres, Laskar, Makazaga, Murua, 2013 



\subsection{Format du fichier nf\_initext}

Ce fichier contient les conditions initiales (masses et coordonn\'ees) des syst\`emes plan\'etaires. 
Ce fichier stocke un syst\`eme plan\'etaire par ligne.

Les masses sont exprimées en masse solaire.  La masse solaire de référence dépend du flag ref\_gmsun.
Les unit\'es des coordonn\'ees des plan\`etes doivent \^etre en UA, an et radians.

Sur chaque ligne, on a :
\begin{itemize}
\item colonne 1 : chaine sans espace donnant le nom du syst\`eme. Par exemple P0001 ou N0002, .... .
\item colonne 2 : nombre de plan\`etes (sans l'\'etoile) , nomm\'e nbplan.
\item colonne 3 : Masse de  l'\'etoile exprimée en masse solaire (=1 pour le système solaire)
\item colonne 4 \`a 4+nbplan-1 : Masse des plan\`etes exprimée en masse solaire 
\item colonne 4+nbplan : type de coordonn\'ees initiales des plan\`etes
\begin{itemize}
\item 1:  elliptiques h\'eliocentriques canoniques
	     CI(1:6) = (a,e,I,M,omega,Omega)
\item 2:  elliptiques h\'eliocentriques non canoniques
	     CI(1:6) = (a,e,I,M,omega,Omega)
\item 3:  elliptiques h\'eliocentriques canoniques
	     CI(1:6) = (a,la,k,h,q,p)
\item 4:  elliptiques h\'eliocentriques non canoniques
	     CI(1:6) = (a,la,k,h,q,p)
\item 5:  positions vitesses h\'eliocentriques
	     CI(1:6) = (x,y,z,vx,vy,vz)
\end{itemize}

\item colonne 4+nbplan+1 \`a  4+nbplan+6 :   coordonn\'ees initiales (6 composantes) de la plan\`ete 1
\item  colonnes suivantes :   coordonn\'ees initiales (6 composantes) pour les plan\`etes suivantes
\end{itemize}

 Par exemple, si on a 3 plan\`etes avec des positions/vitesses h\'eliocentriques, on a dans les colonnes :
 
\begin{tabular}{|c|c|c|c|c|c|c|c|c|c|} \hline
1 &  2 &  3 & 4 & 5 &6 &7 &8-13 &14-19 &20-25 \\ \hline
P0001 & 3 & $M_{star}$  & $M_1$ &  $M_2$ & $M_3$  &5 &$CI_1(1:6)$ & $CI_2(1:6)$&$CI_3(1:6)$\\    \hline
\end{tabular}


\subsection{Format du fichier ctrl\_distpla\_nfdistmin }

Ce fichier contient les distances minimales entre les plan\`etes des syst\`emes plan\'etaires. 
Ce fichier stocke un syst\`eme plan\'etaire par ligne. L'ordre des syst\`emes plan\'etaires doit \^etre le m\^eme que dans le fichier nf\_initext.

L'unit\'e des distances minimales doivent \^etre en UA.

Sur chaque ligne, on a :
\begin{itemize}
\item colonne 1 : chaine sans espace donnant le nom du syst\`eme. Par exemple P0001 ou N0002, .... .
\item colonne 2 : nombre de plan\`etes (sans l'\'etoile) , nomm\'e nbplan.
\item colonne 3 \`a 3+nbplan-1 : Distance minimale pour chaque plan\`ete. 
\end{itemize}

 Par exemple, si on a 3 plan\`etes, on a dans les colonnes :
 
\begin{tabular}{|c|c|c|c|c|} \hline
1 &  2 &  3 & 4 & 5  \\ \hline
P0001 & 3 & $dmin_{1}$  & $dmin_{2}$ & $dmin_{3}$\\    \hline
\end{tabular}


%%%%%%%%%%%%%%%%%%%%%%%
%%%%%%%%%%%%%%%%%%%%%%%
%%%%%%%%%%%%%%%%%%%%%%%
\section{Fichiers de sortie}

%%%%%%%%%%%%%%%%%%%%%%%
\subsection{Format du fichier {\bf ???.ci} }

Ce fichier contient les conditions initiales (masses et coordonn\'ees) des syst\`emes plan\'etaires. 
Ce fichier stocke un syst\`eme plan\'etaire par ligne.

Son format est identique \`a   celui de nf\_initext.

%%%%%%%%%%%%%%%%%%%%%%%
\subsection{Format du fichier {\bf ???.control} }

Ce fichier contient 5 colonnes et indique pour pour chaque condition initiale si l'int\'egration s'est bien d\'eroul\'ee ou non.
Ce fichier stocke un syst\`eme plan\'etaire par ligne.

Sur chaque ligne, on a :
\begin{itemize}
\item colonne 1 : chaine sans espace donnant le nom du syst\`eme. Par exemple P0001 ou N0002, .... .
\item colonne 2 : 
\begin{itemize}
\item 0: l'int\'egration s'est correctement termin\'ee
\item -3: probl\`eme de convergence dans kepsaut. L'int\'egration s'est interrompue.
\item -4: cas non elliptique.  L'int\'egration s'est interrompue.
\item -5: variation trop grande de l'énergie.  La colonne 6 contient la valeur absolue de l'erreur relative de l'énergie par rapport \`a l'énergie au temps 0. L'int\'egration s'est interrompue.
\item -6: corps trop proche de l'\'etoile.  La colonne 6 contient la distance de la plan\`ete \`a l'\'etoile. L'int\'egration s'est interrompue.
\item -7: corps trop loin de l'\'etoile.  La colonne 6 contient la distance de la plan\`ete \`a l'\'etoile. L'int\'egration s'est interrompue.
\item -9: corps trop proche d'une plan\`ete.  La colonne 6 contient le num\'ero de la plan\`ete qui est trop proche. L'int\'egration s'est interrompue.
\end{itemize}
\item colonne 3 : temps initial de l'int\'egration
\item colonne 4 : temps finale de l'int\'egration
\item colonne 5 : corps (si disponible) ayant g\'en\'er\'e l'erreur
\item colonne 6 : 0 si aucune erreur. Sinon, elle contient une valeur d\'ependante de la colonne 2.
\end{itemize}


%%%%%%%%%%%%%%%%%%%%%%%
\subsection{Format du fichier {\bf ???.int} }

Chaque fichier contient un seul syst\`eme plan\'etaire.
Ce fichier contient 5 colonnes et stocke la valeur des int\'egrales premi\`eres  : \'energie et moment cin\'etique.

Sur chaque ligne, on a : 

\begin{tabular}{|c|c|c|} \hline
colonne 1 &  colonne 2 & colonne 3-5 \\ \hline
temps & \'energie & moment cin\'etique (x,y,z)\\    \hline
\end{tabular}

La premi\`ere ligne contient la valeur initiale des int\'egrales premi\`eres. Les lignes suivantes contient la différence (absolue) des intégrales par rapport à la valeur initiale.


%%%%%%%%%%%%%%%%%%%%%%%
\subsection{Format du fichier {\bf ???.car} }

Ce fichier contient les positions h\'eliocentriques et vitesses h\'eliocentriques cart\'esiennes des plan\`etes. Les unit\'es sont en AU et AU/an.
Chaque fichier contient un seul syst\`eme plan\'etaire.


Sur chaque ligne, on a : 

\begin{tabular}{|c|c|c|c|} \hline
colonne 1 &   colonne 2-7 & colonne 8-13 & ... \\ \hline
temps & (x,y,z,vx,vy,vz) de la plan\`ete 1  & (x,y,z,vx,vy,vz) de la plan\`ete 2 & ... \\    \hline
\end{tabular}

%%%%%%%%%%%%%%%%%%%%%%%
\subsection{Format du fichier {\bf ???.ell} }

Ce fichier contient les \'el\'ements elliptiques des plan\`etes. Le type d'\'el\'ement d\'epend du param\`etres  {\bf out\_ell}. Les unit\'es sont en AU, an et radians.
Chaque fichier contient un seul syst\`eme plan\'etaire.

Sur chaque ligne, on a : 

\begin{tabular}{|c|c|c|c|} \hline
colonne 1 &   colonne 2-7 & colonne 8-13 & ... \\ \hline
temps & ell(1:6) de la plan\`ete 1  & ell(1:6) de la plan\`ete 2 & ... \\    \hline
\end{tabular}


%%%%%%%%%%%%%%%%%%%%%%%
\subsection{Format du fichier  {\bf ???.ctrlstar\_car}}

Ce fichier contient les positions h\'eliocentriques et vitesses h\'eliocentriques cart\'esiennes des plan\`etes lors de l'arr\^et de l'int\'egration d\^u \`a une distance trop proche ou lointaine \`a l'\'etoile. Les unit\'es sont en AU et AU/an.

 Il y a un fichier par processeur. Chaque fichier contient plusieurs syst\`emes plan\'etaires.


Sur chaque ligne, on a : 

\begin{tabular}{|c|c|c|c|c|} \hline
colonne 1 &   colonne 2 & colonne 3-8 & colonne 9-14 & ... \\ \hline
nom & temps & (x,y,z,vx,vy,vz) de la plan\`ete 1  & (x,y,z,vx,vy,vz) de la plan\`ete 2 & ... \\    \hline
\end{tabular}

%%%%%%%%%%%%%%%%%%%%%%%
\subsection{Format du fichier {\bf ???.ctrlstar\_ell}}

Ce fichier contient les \'el\'ements elliptiques des plan\`etes lors de l'arr\^et de l'int\'egration d\^u \`a une distance trop proche ou lointaine \`a l'\'etoile. Le type d'\'el\'ement d\'epend du param\`etres  {\bf out\_ell}. Les unit\'es sont en AU, an et radians.

 Il y a un fichier par processeur. Chaque fichier contient plusieurs syst\`emes plan\'etaires.

Sur chaque ligne, on a : 

\begin{tabular}{|c|c|c|c|c|} \hline
colonne 1 &   colonne 2 & colonne 3-8 & colonne 9-14 & ... \\ \hline
nom & temps & ell(1:6) de la plan\`ete 1  & ell(1:6) de la plan\`ete 2 & ... \\    \hline
\end{tabular}

%%%%%%%%%%%%%%%%%%%%%%%
\subsection{Format du fichier  {\bf ???.ctrlpla\_car}}

Ce fichier contient les positions h\'eliocentriques et vitesses h\'eliocentriques cart\'esiennes des plan\`etes lors de l'arr\^et de l'int\'egration d\^u \`a une distance trop proche entre plan\`etes. Les unit\'es sont en AU et AU/an.

 Il y a un fichier par processeur. Chaque fichier contient plusieurs syst\`emes plan\'etaires.


Sur chaque ligne, on a : 

\begin{tabular}{|c|c|c|c|c|} \hline
colonne 1 &   colonne 2 & colonne 3-8 & colonne 9-14 & ... \\ \hline
nom & temps & (x,y,z,vx,vy,vz) de la plan\`ete 1  & (x,y,z,vx,vy,vz) de la plan\`ete 2 & ... \\    \hline
\end{tabular}

%%%%%%%%%%%%%%%%%%%%%%%
\subsection{Format du fichier  {\bf ???.ctrlpla\_ell}}

Ce fichier contient les \'el\'ements elliptiques des plan\`etes lors de l'arr\^et de l'int\'egration d\^u \`a une distance trop proche entre plan\`etes. Le type d'\'el\'ement d\'epend du param\`etres  {\bf out\_ell}. Les unit\'es sont en AU, an et radians.

 Il y a un fichier par processeur. Chaque fichier contient plusieurs syst\`emes plan\'etaires.

Sur chaque ligne, on a : 

\begin{tabular}{|c|c|c|c|c|} \hline
colonne 1 &   colonne 2 & colonne 3-8 & colonne 9-14 & ... \\ \hline
nom & temps & ell(1:6) de la plan\`ete 1  & ell(1:6) de la plan\`ete 2 & ... \\    \hline
\end{tabular}

%%%%%%%%%%%%%%%%%%%%%%%
\subsection{Format du fichier {\bf ???.minmax\_aei} }

Ce fichier contient les minimum, maximum et moyenne en a,e et i sur une tranche de temps. Les unit\'es sont en AU et radians.
Les types des \'el\'ements elliptiques d\'ependent du param\`etre  {\bf minmax\_aei\_elltype}.

 Il y a un fichier par processeur. Chaque fichier contient plusieurs syst\`emes plan\'etaires.
 
 
Sur chaque ligne, on a dans chaque colonne: 

\begin{tabular}{|c|c|c|c|c|c|c|c|c|c|c|c|c|} \hline
 1 &    2 &  \multicolumn{9}{c|}{3-11} & \multicolumn{2}{c|}{12-...} \\ \hline
  &    &  \multicolumn{9}{c|}{plan\`ete 1} &  \multicolumn{2}{c|}{plan\`ete 2 ...} \\ \hline
  &    &  \multicolumn{3}{c|}{a} & \multicolumn{3}{c|}{e} & \multicolumn{3}{c|}{i} & \multicolumn{2}{c|}{a ....} \\ \hline
nom & temps &  min   &moy & max  &  min   & moy & max   & min   & moy & max &  min &... \\  
 & final &     & &   &     &  &    &    &  &  &   &
\\\hline
\end{tabular}
 Ici, le temps final est le temps de fin de chaque tranche. Le fichier contient toutes les tranches d'une m\^eme  condition initiale.

%%%%%%%%%%%%%%%%%%%%%%%
\subsection{Format du fichier {\bf ???.minmax\_alp} }

Ce fichier contient les minimum, maximum et moyenne en $a_{p(1)}-a_{p(2)}$, $\lambda_{p(1)}-\lambda_{p(2)}$ et $\varpi_{p(1)}-\varpi_{p(2)}$ sur une tranche de temps. Les unit\'es sont en AU et radians. Pour les diff\'erences d'angle, il y a une double d\'etermination entre $[0,2\pi]$ et entre $[-\pi, \pi]$.
Les types des \'el\'ements elliptiques d\'ependent du param\`etre  {\bf minmax\_alp\_elltype}.

 Il y a un fichier par processeur. Chaque fichier contient plusieurs syst\`emes plan\'etaires. Le fichier contient toutes les tranches d'une m\^eme  condition initiale.


Sur chaque ligne, on a dans chaque colonne: 

\begin{tabularx}{\textwidth}{|l|l|X|}
 \hline
 colonne &      \multicolumn{2}{c|}{description} \\ \hline
1  &    \multicolumn{2}{l|}{nom} \\ \hline
2  &    \multicolumn{2}{l|}{temps final de chaque tranche} \\ \hline
3 &    & min\\
4 &  $a_{p(1)}-a_{p(2)}$ & moy\\
5 & &   max\\ \hline
6 & &   min\\
7 & $\lambda_{p(1)}-\lambda_{p(2)}$ sur $[0,2\pi]$&moy\\
8 &    & max\\ \hline
9 &    & min\\
10& $\lambda_{p(1)}-\lambda_{p(2)}$ sur $[-\pi,\pi]$ & moy\\
11 & &    max\\ \hline
12 & &    min\\
13 &   $\varpi_{p(1)}-\varpi_{p(2)}$ sur $[0,2\pi]$ & moy\\
14 & &    max\\ \hline
15 & &    min\\
16 & $\varpi_{p(1)}-\varpi_{p(2)}$ sur $[-\pi,\pi]$ & moy\\
17 & &    max\\ \hline
\end{tabularx}

%%%%%%%%%%%%%%%%%%%%%%%
\subsection{Format du fichier {\bf ???.minmax\_alc} }

Ce fichier contient les minimum, maximum et moyenne en $a_{p(1)}-a_{p(2)}$, $\lambda_{p(1)}-\lambda_{p(2)}$ et $\varpi_{p(1)}-\varpi_{p(2)}$ sur une tranche de temps. Les unit\'es sont en AU et radians. Pour les diff\'erences d'angle,les différences d'angles sotn redressés pour obtenir une différence continue.
Les types des \'el\'ements elliptiques d\'ependent du param\`etre  {\bf minmax\_alc\_elltype}.

 Il y a un fichier par processeur. Chaque fichier contient plusieurs syst\`emes plan\'etaires. Le fichier contient toutes les tranches d'une m\^eme  condition initiale.


Sur chaque ligne, on a dans chaque colonne: 

\begin{tabularx}{\textwidth}{|l|l|X|}
 \hline
 colonne &      \multicolumn{2}{c|}{description} \\ \hline
1  &    \multicolumn{2}{l|}{nom} \\ \hline
2  &    \multicolumn{2}{l|}{temps final de chaque tranche} \\ \hline
3 &    & min\\
4 &  $a_{p(1)}-a_{p(2)}$ & moy\\
5 & &   max\\ \hline
6 & &   min\\
7 & $\lambda_{p(1)}-\lambda_{p(2)}$ &moy\\
8 &    & max\\ \hline
12 & &    min\\
13 &   $\varpi_{p(1)}-\varpi_{p(2)}$& moy\\
14 & &    max\\ \hline
\end{tabularx}

%%%%%%%%%%%%%%%%%%%%%%%
\subsection{Format du fichier {\bf ???.minmax\_ae2} }

Ce fichier contient les minimum, maximum et moyenne en $(a_{p(1)}-a_{p(2)})^2$, $(e_{p(1)}-e_{p(2)})^2$ et $(a_{p(1)}-a_{p(2)})^2+(e_{p(1)}-e_{p(2)})^2$ sur une tranche de temps. Les unit\'es sont en AU.
Les types des \'el\'ements elliptiques d\'ependent du param\`etre  {\bf minmax\_ae2\_elltype}.

 Il y a un fichier par processeur. Chaque fichier contient plusieurs syst\`emes plan\'etaires. Le fichier contient toutes les tranches d'une m\^eme  condition initiale.


Sur chaque ligne, on a dans chaque colonne: 

\begin{tabularx}{\textwidth}{|l|l|X|}
 \hline
 colonne &      \multicolumn{2}{c|}{description} \\ \hline
1  &    \multicolumn{2}{l|}{nom} \\ \hline
2  &    \multicolumn{2}{l|}{temps final de chaque tranche} \\ \hline
3 &    & min\\
4 &  $(a_{p(1)}-a_{p(2)})^2$ & moy\\
5 & &   max\\ \hline
3 &    & min\\
4 &  $(e_{p(1)}-e_{p(2)})^2$ & moy\\
5 & &   max\\ \hline
3 &    & min\\
4 &  $(a_{p(1)}-a_{p(2)})^2+(e_{p(1)}-e_{p(2)})^2$ & moy\\
5 & &   max\\ \hline
\end{tabularx}

%%%%%%%%%%%%%%%%%%%%%%%
\subsection{Format du fichier {\bf ???.naf\_alkhqp} }

Ce fichier contient l'analyse en fr\'equence en $a\exp^{\imath\lambda}, k+\imath h, q+\imath p$ sur une tranche de temps. Les unit\'es des fr\'equences d\'ependent de naf\_alkhqp\_dtour.

 Il y a un fichier par processeur. Chaque fichier contient plusieurs syst\`emes plan\'etaires. Le fichier contient toutes les tranches d'une m\^eme  condition initiale.


Sur chaque ligne, on a dans chaque colonne: 

\begin{tabularx}{\textwidth}{|l|l|l|l|X|}
 \hline
 colonne &      \multicolumn{4}{c|}{description} \\ \hline
1  &    \multicolumn{4}{l|}{nom} \\ \hline
2  &    \multicolumn{4}{l|}{temps initial (T0) de chaque tranche} \\ \hline
3 & &   & & fr\'equence\\
4 &planete 1 &$a\exp^{\imath\lambda}$& terme 1 & amplitude (partie r\'eelle)\\
5 & &   & &amplitude (partie imaginaire)\\ \hline
\dots & planete 1 &$a\exp^{\imath\lambda}$& terme ?? &\dots \\ \hline
 & &   & &fr\'equence\\
 &planete 1 & $a\exp^{\imath\lambda}$ & terme $naf\_alkhqp\_nterm$ & amplitude (partie r\'eelle)\\
 & &   & &amplitude (partie imaginaire)\\ \hline
 & &   & & fr\'equence\\
 &planete 1 &$k+\imath h$& terme 1 & amplitude (partie r\'eelle)\\
 & &   & &amplitude (partie imaginaire)\\ \hline
\dots & planete 1 &$k+\imath h$& terme ?? &\dots \\ \hline
 & &   & &fr\'equence\\
 &planete 1 & $k+\imath j$ & terme $naf\_alkhqp\_nterm$ & amplitude (partie r\'eelle)\\
 & &   & &amplitude (partie imaginaire)\\ \hline
& &   & & fr\'equence\\
 &planete 1 &$q+\imath p$& terme 1 & amplitude (partie r\'eelle)\\
 & &   & &amplitude (partie imaginaire)\\ \hline
\dots & planete 1 &$q+\imath p$& terme ?? &\dots \\ \hline
 & &   & &fr\'equence\\
 &planete 1 & $q+\imath p$ & terme $naf\_alkhqp\_nterm$ & amplitude (partie r\'eelle)\\
 & &   & &amplitude (partie imaginaire)\\ \hline
\dots & planete 2 &$a\exp^{\imath\lambda}$& terme 1 &fr\'equence \\ \hline
\dots & & & \dots &\\\hline
\end{tabularx}



%%%%%%%%%%%%%%%%%%%%%%%
\subsection{Format du fichier {\bf ???.naf\_alkh} }

Ce fichier contient l'analyse en fr\'equence en $a\exp^{\imath\lambda}, k+\imath h$ sur une tranche de temps. Les unit\'es des fr\'equences d\'ependent de naf\_alkhqp\_dtour.

 Il y a un fichier par processeur. Chaque fichier contient plusieurs syst\`emes plan\'etaires. Le fichier contient toutes les tranches d'une m\^eme  condition initiale.


Sur chaque ligne, on a dans chaque colonne: 

\begin{tabularx}{\textwidth}{|l|l|l|l|X|}
 \hline
 colonne &      \multicolumn{4}{c|}{description} \\ \hline
1  &    \multicolumn{4}{l|}{nom} \\ \hline
2  &    \multicolumn{4}{l|}{temps initial (T0) de chaque tranche} \\ \hline
3 & &   & & fr\'equence\\
4 &planete 1 &$a\exp^{\imath\lambda}$& terme 1 & amplitude (partie r\'eelle)\\
5 & &   & &amplitude (partie imaginaire)\\ \hline
\dots & planete 1 &$a\exp^{\imath\lambda}$& terme ?? &\dots \\ \hline
 & &   & &fr\'equence\\
 &planete 1 & $a\exp^{\imath\lambda}$ & terme $naf\_alkhqp\_nterm$ & amplitude (partie r\'eelle)\\
 & &   & &amplitude (partie imaginaire)\\ \hline
 & &   & & fr\'equence\\
 &planete 1 &$k+\imath h$& terme 1 & amplitude (partie r\'eelle)\\
 & &   & &amplitude (partie imaginaire)\\ \hline
\dots & planete 1 &$k+\imath h$& terme ?? &\dots \\ \hline
 & &   & &fr\'equence\\
 &planete 1 & $k+\imath j$ & terme $naf\_alkhqp\_nterm$ & amplitude (partie r\'eelle)\\
 & &   & &amplitude (partie imaginaire)\\ \hline
\dots & planete 2 &$a\exp^{\imath\lambda}$& terme 1 &fr\'equence \\ \hline
\dots & & & \dots &\\\hline
\end{tabularx}

%%%%%%%%%%%%%%%%%%%%%%%
\subsection{Format du fichier {\bf ???.naf\_diffalp} }

Ce fichier contient l'analyse en fr\'equence en  $\exp^{\imath(\lambda_{p(1)}-\lambda_{p(2)})}$ et $\exp^{\imath(\varpi_{p(1)}-\varpi_{p(2)})}$ sur une tranche de temps. Les unit\'es des fr\'equences d\'ependent de naf\_diffalp\_dtour.

 Il y a un fichier par processeur. Chaque fichier contient plusieurs syst\`emes plan\'etaires. Le fichier contient toutes les tranches d'une m\^eme  condition initiale.


Sur chaque ligne, on a dans chaque colonne: 

\begin{tabularx}{\textwidth}{|l|l|l|X|}
 \hline
 colonne &      \multicolumn{3}{c|}{description} \\ \hline
1  &    \multicolumn{3}{l|}{nom} \\ \hline
2  &    \multicolumn{3}{l|}{temps initial (T0) de chaque tranche} \\ \hline
3 & &    & fr\'equence\\
4 &$\exp^{\imath(\lambda_{p(1)}-\lambda_{p(2)})}$& terme 1 & amplitude (partie r\'eelle)\\
5 &   & &amplitude (partie imaginaire)\\ \hline
\dots &  $\exp^{\imath(\lambda_{p(1)}-\lambda_{p(2)})}$& terme ?? &\dots \\ \hline
 & &    &fr\'equence\\
 &  $\exp^{\imath(\lambda_{p(1)}-\lambda_{p(2)})}$ & terme $naf\_diffalp\_nterm$ & amplitude (partie r\'eelle)\\
 & &    &amplitude (partie imaginaire)\\ \hline
 & &   & fr\'equence\\
 &$\exp^{\imath(\varpi_{p(1)}-\varpi_{p(2)})}$& terme 1 & amplitude (partie r\'eelle)\\
 &    & &amplitude (partie imaginaire)\\ \hline
\dots & $\exp^{\imath(\varpi_{p(1)}-\varpi_{p(2)})}$& terme ?? &\dots \\ \hline
 & &    &fr\'equence\\
 & $\exp^{\imath(\varpi_{p(1)}-\varpi_{p(2)})}$ & terme $naf\_diffalp\_nterm$ & amplitude (partie r\'eelle)\\
 &   & &amplitude (partie imaginaire)\\ \hline
\end{tabularx}

%%%%%%%%%%%%%%%%%%%%%%%
\subsection{Format du fichier {\bf ???.naf\_difflpm} }

Ce fichier contient l'analyse en fr\'equence en  $(\lambda_{p(1)}-\lambda_{p(2)})$ et $(\varpi_{p(1)}-\varpi_{p(2)})$ sur une tranche de temps. Les unit\'es des fr\'equences d\'ependent de naf\_difflpm\_dtour.

 Il y a un fichier par processeur. Chaque fichier contient plusieurs syst\`emes plan\'etaires. Le fichier contient toutes les tranches d'une m\^eme  condition initiale.


Sur chaque ligne, on a dans chaque colonne: 

\begin{tabularx}{\textwidth}{|l|l|l|X|}
 \hline
 colonne &      \multicolumn{3}{c|}{description} \\ \hline
1  &    \multicolumn{3}{l|}{nom} \\ \hline
2  &    \multicolumn{3}{l|}{temps initial (T0) de chaque tranche} \\ \hline
3 & &    & fr\'equence\\
4 &$(\lambda_{p(1)}-\lambda_{p(2)})$ sur $[0,2\pi]$& terme 1 & amplitude (partie r\'eelle)\\
5 &   & &amplitude (partie imaginaire)\\ \hline
\dots &  $(\lambda_{p(1)}-\lambda_{p(2)})$ sur $[0,2\pi]$& terme ?? &\dots \\ \hline
 & &    &fr\'equence\\
 &  $(\lambda_{p(1)}-\lambda_{p(2)})$ sur $[0,2\pi]$ & terme $naf\_difflpm\_nterm$ & amplitude (partie r\'eelle)\\
 & &    &amplitude (partie imaginaire)\\ \hline
 & &    & fr\'equence\\
 &$(\lambda_{p(1)}-\lambda_{p(2)})$ sur $[-\pi,\pi]$& terme 1 & amplitude (partie r\'eelle)\\
 &   & &amplitude (partie imaginaire)\\ \hline
\dots &  $(\lambda_{p(1)}-\lambda_{p(2)})$ sur $[-\pi,\pi]$& terme ?? &\dots \\ \hline
 & &    &fr\'equence\\
 &  $(\lambda_{p(1)}-\lambda_{p(2)})$ sur $[-\pi,\pi]$ & terme $naf\_difflpm\_nterm$ & amplitude (partie r\'eelle)\\
 & &    &amplitude (partie imaginaire)\\ \hline
 & &   & fr\'equence\\
 &$(\varpi_{p(1)}-\varpi_{p(2)})$ sur $[0,2\pi]$& terme 1 & amplitude (partie r\'eelle)\\
 &    & &amplitude (partie imaginaire)\\ \hline
\dots & $(\varpi_{p(1)}-\varpi_{p(2)})$ sur $[0,2\pi]$& terme ?? &\dots \\ \hline
 & &    &fr\'equence\\
 & $(\varpi_{p(1)}-\varpi_{p(2)})$ sur $[0,2\pi]$ & terme $naf\_difflpm\_nterm$ & amplitude (partie r\'eelle)\\
 &   & &amplitude (partie imaginaire)\\ \hline
 & &   & fr\'equence\\
 &$(\varpi_{p(1)}-\varpi_{p(2)})$ sur $[-\pi,\pi]$& terme 1 & amplitude (partie r\'eelle)\\
 &    & &amplitude (partie imaginaire)\\ \hline
\dots & $(\varpi_{p(1)}-\varpi_{p(2)})$ sur $[-\pi,\pi]$& terme ?? &\dots \\ \hline
 & &    &fr\'equence\\
 & $(\varpi_{p(1)}-\varpi_{p(2)})$ sur $[-\pi,\pi]$ & terme $naf\_difflpm\_nterm$ & amplitude (partie r\'eelle)\\
 &   & &amplitude (partie imaginaire)\\ \hline
\end{tabularx}

\end{document}  